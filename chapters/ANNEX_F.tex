\appendix{Countdown buttons Tutorial}

This tutorial is for adding dynamic countdowns for the elements in a Spinner
object. In the particular case of this application, the countdown is activated
based when the 'Shoot' button is pressed.\newline
\begin{enumerate}
  \item \textbf{Somewhere in the code, a ThreadPoolExecutor has to be created}:
  
\begin{verbatim}

private static final BlockingQueue<Runnable> workQueue = 
										new LinkedBlockingQueue<Runnable>(10);
ThreadPoolExecutor executor = 
				new ThreadPoolExecutor(3, 10, 3, TimeUnit.SECONDS, workQueue);
\end{verbatim}
	
  \item \textbf{Then, we define the class that implements Runnable. We will call
  it WeaponReportingTask}:

\begin{verbatim}
public class WeaponReportingTask extends AsyncTask<Weapon, Integer, Integer> {	
  		
	private boolean hasFinished = true;
	
	ArrayList<String> weaponsArray = null;
	ArrayAdapter<String> dataAdapter = null;
	int selectedIndex = 0;
	String initialWeaponName = "";	
	
	public WeaponReportingTask(
						ArrayList<String> newWeaponsArray, 
						ArrayAdapter<String> newDataAdapter,
						int newSelectedIndex
						){		
		super();
		
		hasFinished = true;
		
		weaponsArray = newWeaponsArray;
		dataAdapter = newDataAdapter;
		selectedIndex = newSelectedIndex;
		
		initialWeaponName = weaponsArray.get(selectedIndex);
	}
	
	

	@Override
 	protected Integer doInBackground(Weapon... params) {
	
		hasFinished = false;
		
		Weapon weapon = params[0];
 		if(weapon.getLastUsed() == null) 
 			return 0;
 		
 		int secondsElapsed = 
 					(int) MapUtils.getTimeDelta(weapon.getLastUsed(), true);
 		
 		while(secondsElapsed <= weapon.cooldown){
 			
 			publishProgress(weapon.cooldown - secondsElapsed);
 			
 			try {
				Thread.sleep(200);
			} catch (InterruptedException e) {
				// TODO Auto-generated catch block
				e.printStackTrace();
			}
 			
 			secondsElapsed = 
 					(int) MapUtils.getTimeDelta(weapon.getLastUsed(), true);
 		}
 		
 		return 1;
 	}
     
    
    protected synchronized void onProgressUpdate(Integer... progress) {
    	
    	if(progress != null)
    		weaponsArray.set(
    			selectedIndex, 
    			initialWeaponName+" ("+progress[0].toString()+" )"
    			);
    	
    	else 
    		weaponsArray.set(
    			selectedIndex, 
    			initialWeaponName
    			);
    	
    	dataAdapter.notifyDataSetChanged();
    }

    
    @Override
    protected void onPostExecute(Integer result) {
    	weaponsArray.set(selectedIndex, initialWeaponName);
    	dataAdapter.notifyDataSetChanged();    	
    	super.onPostExecute(result);
    	
    	hasFinished = true;
    	
    }
    
    @Override
    protected void onCancelled() {
    	weaponsArray.set(selectedIndex, initialWeaponName);
    	dataAdapter.notifyDataSetChanged();
    	super.onCancelled();
    	
    	hasFinished = true;
    }



	public boolean hasFinished() {
		return hasFinished;
	}
    
}

\end{verbatim}

  \item \textbf{And then, on 'Shoot', we launch it}:

\begin{verbatim}
WeaponReportingTask cooldownTimer = 
					new WeaponReportingTask(
							weapons,
							(ArrayAdapter<String>)weaponSpinner.getAdapter(),
							currentWeaponIndex
							);
cooldownTimer.executeOnExecutor(
		executor,
		currentWeapon //this object is used in the WeaponReportingTask only  
					  //for the cooldown time specific to the Weapon object.
		);
\end{verbatim}


\end{enumerate}

