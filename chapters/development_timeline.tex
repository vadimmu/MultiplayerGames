\section{Development timeline}

This section will follow the development of the application. The back end and
mechanics of the game will be analyzed and presented in their evolution,
also briefly describing aspects of the front end. \newline

The evolution of the server and the client will be presented chronologically.

\subsection{The exploration phase}

The development of the app has started on the 14th of January 2013. Everything
described from now on is a process of creativity and learning, intertwined
together. But before the development came the idea, which took three months to
concoct.\newline


\subsubsection{Creating the game concept}

The project has started as a proposal to add a multiplayer feature to a tour
app. It was hazy and unstructured. The main point was that the tour app was
targeting individual users. But then, what happens when a larger group tours
together? At the current stage of development, the tour app was offering an
individual experience within the group. So the proposal of this paper's author
was that the group experiences the tour together, somehow. \newline

Simply adding the functionality to see all the others within the group on the
map was not sufficient - it would just help if someone got lost or went astray.
Otherwise, it was concluded that the user experience would not be improved in
any significant way. Then came the idea of adding small games, hidden caches or
quests and so on. The best idea that still had the tour app as a main platform
was to add detours from the main track as bonuses. On those detours, the people
would have to solve various riddles and small puzzles to get points and find out
about hidden historical spots or interesting facs about the places they are
visiting. \newline

At this point, contoured the following addon to the main app was contoured: the
players would have a main tour path and, as they passed by certain waypoints,
would be offered to go through a bonus/extra track within the area that they are
visiting. If enough of them would agree to do this(by an in-app vote, for
example), they would be presented with a new set of waypoints. These waypoints
would belong to a number of categories: normal waypoints, waypoints where they
would have to split, waypoints where they would have to be together and
waypoints where the whole group would have to wait within a virtual circle,
while one delegate(through vote) would find an item or solve a riddle.\newline

Another scenario has been proposed, during the research: In the case of the
group waiting and one member being delegated to solve a riddle or find an item,
a means of cooperation can be brought up: When the team votes and chooses the
delegate, his screen turns black(no map) and the rest of the team can see
the goal on their phones. Through messages or VoIP, they would help the delegate
navigate to the goal. Then, once the goal has been found or reached, the virtual
circle would disappear and the whole team would be free to move. If anyone would
exit the circle at some point, they would receive penalties and eventually get
disconnected from the game.\newline

Gradually, the ideas for multiplayer features went astray from the tour app,
towards multiplayer GPS-based games. The author has explored the idea of
GPS-based puzzle / adventure games and encountered ARIS, a platform for creating
such games. Then Tourality was discovered, and WarFinger GPS, which have been
the main sources of inspiration for the 'War Game' and a few other games that
didn't make it in the main concept, but are to be developed within the game as
future work.\newline

There has been a point when all the GPS-based mobile games that could be
found were simply tried out and evaluated. At this point, the goals for this
app were already in mind: It should move the players to an outdoor environment
and have them walk and run as the main activity, while using the game itself for
an improved experience. Hide and seek and Tag were considered as a model of
entertaining game to be played by a group. The games found and already
enumerated had,by the subjective opinion of the author, one of two major
disadvantages:
\begin{enumerate}
  \item Did not engage the users enough: Games such as Parallel Mafia, SCVNGR or
  Please Stay Calm do not motivate the player to move around much. They also
  offer a very dim user experience in areas with few or no players.
  \item Engaged the users too much: Mobile users, even hardcore gamers, do not
  spend too much time playing on the phone. Rather, they would play on consoles
  or computers, for better immersion. No matter how good the game is, it is
  still displayed on a small phone screen(tablets are not considered in this
  paper). Games such as Ingress and Shadow Cities offer a better and more
  immersive story, but are still demanding of the player and request the full
  focus of the phone owner. This was considered to be a major downside, as
  at least some people(the author included), when they get out of their home,
  prefer to focus on what happens around them  nd only rarely have the time and
  will to play such a demanding game while on the go.
  \item Did not motivate the players to move enough - Only Tourality does not
  possess this downside, as it its various game modes are specifically designed
  for running.
  \item Are location-dependent - All adventure / puzzle games such as those
  developed with ARIS, most MMOs such as Parallel Mafia do require the player to
  be in certain places in order to progress through the game. This means that
  the player is put in one of two situations : he 1. has to get out of
  one's way in order to make any progress within the game, or 2. has to travel
  to remote locations in order to play the game in the first place. This might
  be interesting for some, but will certainly fail to capture the attention of a
  worker or student during weekdays.
\end{enumerate}

The engaging problem can also be linked to a time problem. Games that are too
engaging also require a lot of time to be played. The personal take of the
author on this is that the amount of time dedicated to the mobile game should be
decided by the player and not by the game. \newline

\subsubsection{The game concept}
The app developed is temporarily called People With Guns. It is a GPS-based RTS
/ Shooter hybrid concept protoype for Android. It can played by several people
(The upper limit has not been established yet. Until now, the highest number of
players in the game has been six) that choose to join one of two opposing teams.
The purpose of the game is to use 'weapons' and 'powerups' to defeat the
opposing team. By 'defeat', we mean to use the available tools provided by the
game to reduce the virtual 'health' attribute of all the opponents to zero. 
The current 'tools' are as follows :\newline

The weapons : 
\begin{enumerate}
  \item Pistol
  \item Rifle
  \item Sniper Rifle
  \item Knife   
\end{enumerate}

The 'powerups' :
\begin{enumerate}
  \item Invisiblity Cloak
  \item Shield
  \item Painkillers / Heal
\end{enumerate}

Each item used by a player has the following attributes : 
\begin{enumerate}
  \item Cooldown - The amount of time that has to pass until the weapon can be
  used again.
  \item Duration - The amount of time during which a powerup is in effect 
  \item Damage - The amount of health points that are subtracted upon a hit (or
  added, in the case of the Painkillers) from the target's total available
  health points.
  \item Range - The maximum distance within which a weapon can be fired.
\end{enumerate}

Based on these attributes, we can make a differentiation between weapons and
powerups. Until this point of the game development, weapons have damage and no
duration. The only powerup that also has damage are the Painkillers - they deal
negative damage to the target. The other two powerups, Invisibility Cloak and
Shield, have a greater than zero 'duration' attribute, but no 'damage'.\newline

Some of the attributes of the weapons and powerups will be modified, during a
phase of balancing. Therefore, their conceptual construction will be described: 

The weapons : 
\begin{enumerate}
  \item Pistol - Weapon with low damage and medium fire rate. All the character
  types have it. It is the basic and least effective weapon of them all. The
  range is reduced.
  \item Rifle - Weapon with low damage and high fire rate. The damage is higher
  than that of the pistol and the cooldown takes half as long. The range is
  reduced the same as that of the Pistol
  \item Sniper Rifle - High damage weapon with very low fire rate. The range is
  far greater than that of the Rifle and Pistol.
  \item Knife - The weapon that deals the highest damage of all. The cooldown is
  greater than that of the Sniper Rifle and the range is very small. 
  \item Invisiblity Cloak - Powerup that makes the player disappear from the
  map for a short while. While the player is invisible, he cannot be shot.
  The effect duration is short, while the cooldown is lengthy.
  \item Shield - Powerup that reduces the damage taken to half, while it is in
  effect. The duration of this powerup is short and the cooldown is lenghty.
  \item Painkillers / Heal - Powerup that functions like a weapon that deals
  negative damage to the player or his/her friends. It's effective immediately
  and requires a long cooldown time.
\end{enumerate}

For more complexity in the game, a number of so-called 'character types' have
been created, from which players can choose. Thus far, there are four character
types implemented in the game: Marine, Medic, Sniper, Scout. Each of these has a
different number of health points and different weapons. Because the actual
health amount will vary during a phase of character balancing, the conceptual
construction of the characters will be mentioned :
\begin{enumerate}
  \item Marine - Has average health and two weapons: Pistol and Rifle.
  Represents the basic attack unit.
  \item Medic - Support unit with Shield and Painkillers/Heal abilities. Has
  average health. 
  \item Sniper - Attack/ defense unit. Has a Sniper Rilfe for shooting at large
  distances. Has low health.
  \item Scout - Attack unit specialized at sneaking up on the victim. Has the
  'Invisibility Cloak' ability for disappearing from the map and the 'Knife'
  weapon for dealing large amounts of damage within a very small range. Has
  large health.
\end{enumerate}

Another character has been created for testing purposes. It has been called the
'All Encompassing' and posesses all the weapons and skills presently existing in
the game and very large health. This character has inadvertently opened a window
for two more game types: 
\begin{enumerate}
  \item 'David versus Goliath' - This be a game of many players using regular
  character types versus a much smaller number of 'All Encompassing' characters.
  \item 'Duel' - During the many small tests made to this app beyond the public
  ones, a new style of playing this game has emerged: in the absence of a large
  enough number of players, two can play in the 'Duel' mode : both use the 'All
  Encompassing' character type and, instead of running around, attempt to win
  the game by optimizing combinations of weapons and powerups. This game is
  generally played side by side, for at most a few minutes and has proved to be
  entertaining for the ones who tried it out.
\end{enumerate}



\subsection{The first development phase}

The first development phase was mainly one of searching for the right tools to
get the job done and testing their functionality.\newline

The initial idea was to use Websockets for communication, a NodeJS-based server
and, after some evaluation, native Android. The messages exchanged between
server and clients would be in the JSON or XML format. Based on some brief
research and the need for this app, JSON was chosen - it is easier to use and it
takes less bytes to transmit the same data as it would with XML. The plan was
also to use simple data structures for the messages that were to be exchanged
between server and clients. For JSON serializing and deserializing, the choices
found viable have been Jackson and Google GSON. After determining their speed
and ease of use Jackson has been chosen, as besides its speed, it offers an easy
way of serializing and deserializing JSON directly into a hierarchy of HashMaps
- a method which was personally preferred by the author for a testing phase.
Mapping something yet unknown into POJOs would have been a much harder task on
the long run of prototype development. \newline

Exploring the use of Websockets has proven unfruitful, as there have been dead
ends :
\begin{enumerate}
  \item The only freely available Websocket library for Android at the time of
  research was Autobahn. The Websocket libraries available for NodeJS were
  Socket.IO, Websocket-Node and ws. None of them worked with Autobahn for
  Android. A forum post was eventually found, in which one of the developers of
  Autobahn stated the reasons for the incompatibility between Socket.IO and
  Autobahn for Android: First, the protocol implementations were based on
  different draft versions. Second, Socket.IO used an HTTP handshake for the
  connection - and that was not supported by Autobahn. The same issue applied to
  Websocket-Node and ws. A Python implementation of Autobahn has been tried for
  the server, but after a few unfruitful attempts, the decision came to use
  Java. In the case of Java, Autobahn for Android does not work. One of the
  reasons: Autobahn subclasses a Handler object that is part of the Android
  SDK. The realization came that until Websockets receives its final version,
  one cannot fully rely on the protocol. And for the purpose of a prototype
  application where communication should not be overly complicated, TCP was
  chosen.
  
  \item Because of the Websocket issue, but also the subjectively perceived slow
  development pace with NodeJS (Libraries are not documented, autocomplete
  mostly does not work and there is no javadoc equivalent), it has been decided
  to switch to Java. The combination of Python  and Websockets was also
  attempted in the meantime, but productivity was perceived as low for the same
  reasons as for NodeJS.
\end{enumerate}

At the end of these attempts and searches, the tools that the author has decided
upon were native Android clients, Java server and TCP communication in between
with JSON messages.\newline

Now the tools were readily available and the development of the app has started.
The first phase development has been split in three sub-phases(sub-components) :
\begin{enumerate}
  \item A game UI that would add some mock players on the map and provide some
  usability insight.
  \item The server
  \item The whole app, based on the initial UI experience.
\end{enumerate}

In developing the initial game UI, three buttons have been added(one for
generating a number of markers ('Generate Markers') on the map, one for further
use('Check Info') and one for shooting ('Shoot')) and a spinner, on top of a
MapView. The spinner served as a weapon selector. Once a weapon was selected (on
launch, the first one in the list was automatically pre-selected), its range
would be drawn on the map. The three buttons have been placed in three corners
of the screen- bottom-left, bottom-right, top-left. The spinner has been layed
to the right of the 'Shoot' button. The functionalities added were as followed :

\begin{enumerate}
  \item 'Shoot' button - Mock method to display a Toast message stating if the
  shot was performed or not and the selected weapon.
  
  \item 'Generate Markers' button - Mock method to randomly generate markers on
  the map (for this game's purposes).
  
  \item 'Check Info' button - sporadic functionality to display a Toast message
  with various info on data structures in the back end.
  
  \item Spinner - selecting weapons to 'shoot'. 
\end{enumerate}


\subsubsection{The structure of the server}
 
\begin{enumerate}
  
  \item A 'communication' module for dealing with adding and managing
  connections and messaging.
  
  \item A 'messages' module for managing the incoming and outgoing messages for
  each connection.
  
  \item A 'game' module for handling in-game data, such as keeping track of
  connected players, teams and game status. 
  
\end{enumerate}

The structure of the 'communication' module:

\begin{enumerate}
  
  \item The 'TcpServer' class that starts listening for incoming TCP connections
  on a given port. 
	
  \item A 'ConnectionManager' class that provides static synchronized methods
  for managing adding and removing connections, MessageSender, MessageReceiver
  and connection keepalive heartbeats.
  
\end{enumerate}

The structure of the 'messages' module:

\begin{enumerate}
  
  \item A 'Messages' class that contains a number of inner classes for
  categorizing the different types of messages.
  
  \item A 'MessageSender' class that deals with sending messages to a single
  client, but can also access all other instances of this class to multicast
  and/or broadcast messages to all the other clients, when needed.
  
  \item A 'MessageReceiver' class that deals with receiving messages from a
  single client
  
  \item A 'HeartbeatListener' class that listens for heartbeats from the
  connected clients and keeps the connections alive or closes them.
  
\end{enumerate}

The structure of the 'game' module:

\begin{enumerate}
  
  \item A 'Player' class that stores information about the players (eg. name,
  'profession', health points and position).
  
  \item A 'GameManager' class that manages the addition, removal and management
  of players.
  
\end{enumerate}

\subsubsection{The Server}

Once started, the server listens on a port. When a remote client connects, the
server adds an InetAddress instance containing the IP address of the client.
Then, a Player object with some default values is initialized and a random UUID
is generated. The Player object is added in one of the HashMaps for the two
teams - home team or away team - using the UUID as key. A MessageReceiver and a
MessageSender are instantiated. The MessageReceiver, MessageSender, a Date
object containing the moment of the connection and the newly generated UUID and
added to HashMaps with the InetAddress of the client as key. The server sends a
configuration json containing the already connected players, the available
professions and their attributes(title, health, weapons, description), the
available weapons and their attributes(name, range, cooldown, duration,
description and usage policy). From now on, whether a player remains connected
to the server depends on the so-called heartbeats : The client will send
periodic heartbeat messages to the server. The server will update the Date
object for the last moment when a heartbeat was received. If a delay beyond a
threshold comes up, the client is removed from the server.\newline

The server now acts only as a relay - game state is held on the client
side.\newline

Because TCP does not allow multicasts and broadcasts and UDP was purposely
avoided, methods for sending multicasts and broadcasts were conceived as
follows: When a client sends a message that requires multicast/broadcast, the
server accesses all the MessageSender objects, iterates through all of them and
sends the given message to all of them. This applies to location updates, shot
alerts and in-game messages.\newline

Once a number of players (1 or more) have connected to the server, they may send
'Ready' messages to the server, stating that they are prepared to enter the
game. The server will check on receipt on each 'Ready' message if all the
players connected to the server are 'Ready'. If yes, a countdown timer will be
started: A broadcast with the seconds left until the game starts will be sent to
all the players. After the countdown, an 'Enter Game' message is broadcast to
all the clients. Because the server holds the information on the positions of
all the players, a condition and another message have been added: when the
distance between the average positions of all the players in each team is at
least some given distance(eg. 500m) and each team member is at most some other
given distance(50m) from the average position of his/her team, 'Start Game'
message is broadcast to all the clients. This will later be used to enable the
weapons for all the players only when they satisfy these conditions.\newline

There is no direct disconnect message between server and client. Disconnection
is done exclusively based on the heartbeat interval.\newline

\subsubsection{The Client}

The client is an Android application that uses one Activity and multiple
Fragments. It does not use any compatibility libraries(as the Google Maps V1 API
requires a MapActivity and the support library requires a FragmentActivity for
fragment use) - therefore, only Android versions equal or higher to
3.0(Honeycomb) are supported. 

The client uses seven fragments for the UI: 'Main Menu', 'Info and
Tutorials', 'Settings', 'Lobby','Lobby Settings', 'Loading' and 'Game':

\begin{enumerate}
  \item 'Main Menu' : It is the entry point of the the application (it is the
  fragment loaded by the Activity after it is created). It features three
  buttons: 'Start Game', 'Settings' and 'Exit'. By pressing 'Start Game', the
  player attempts to join the game. If the connection is successful, he is
  brought into the 'Lobby'. Pressing 'Settings' leads to the homonymous
  screen.
  
  \item 'Settings' contains two TextViews for changing the default IP
  address and port of the server to which the client can connect.
  
  \item 'Loading' is an intermediary fragment that shows a loading widget for
  the duration of time during which the connection to the server is established
  and the client receives the config json from the server. Once this process is
  done, it automatically switches to the 'Lobby' fragment.
  
  \item 'Lobby' is the main game preparation fragment. It shows the lists of
  players in the two teams(nicknames, professions and 'Ready' status). The
  fragment features four buttons: 'Back', 'Change Info', 'Change Team' and
  'Ready'. If the 'Back' button, the connection to the server is closed and the
  'Main Menu' fragment is brought to the front. The 'Change Info' button brings
  upon the 'Lobby Settings' fragment. By pressing the 'Ready' button, the player
  toggles his/her 'Ready' status and sends a message containing this status to
  the server.
  
  \item 'Lobby Settings' is the fragment in which the player can change his/her
  nickname and 'profession'. A textbox for the name, a spinner and a
  description textbox for the profession and an 'Ok' button are provided. 
  
  \item 'Game' is the most important fragment in this app: it provides the UI
  for gameplay. It features a fullscreen MapView that displays the map and, on
  top of it, the 'Shoot' button and weapon selection Spinner. Two other buttons
  are kept on the screen and given various functionalities, according to the
  testing needs.
   
\end{enumerate}

The app usage would go as follows: The player opens the app and is presented the
'Main Menu' UI. If the GPS is not on, a dialog will appear informing him of
this and offering two choices: `Turn GPS on` or 'Exit'. If 'Turn GPS on' is
chosen, the GPS Settings page will be shown and the player can switch it on.
Once this is done, the player clicks 'Start Game' and, after being shortly shown
the 'Loading' fragment, is introduced to the 'Lobby' screen. Here, one can see
that he has been added to one of the teams(Home Team or Away Team) and has been
given a default nickname('Player') and profession('Marine'). This is where he
chooses a team by clicking the 'Change Team' button or opts navigate to the
'Lobby Settings' screen by clicking the 'Change Info' button and change the
'nickname' or 'profession'. Once the player is ready to play, he will press the
'Ready' button. If all the players that are connected to the server are marked
as 'Ready', the server will send a countdown (which is seen on the client side
through Toasts) followed by a 'Start Game' message - at which point the client
app will automatically introduce the 'Game' UI. All the players are shown on the
map by blue(current player), red(enemies) and green(friends) markers. Above each
marker, one can see the nickname, 'profession' and health points of the player.
The selection of players is done by clicking the markers. A selected player will
have his marker drawn in yellow. Selecting a weapon from the provided Spinner
deselects the currently selected player(if there is any) and draws the weapon's
range around the position of the current player. For using the currently
selected 'weapon' or 'powerup' on one of the players, the current player has to
select the target on the screen and press the 'Shoot' button. If the selection
attempt does not satisfy the 'selection policies' provided by the Weapon object,
the map marker will simply not be selected. Otherwise, it will change color to
yellow. If the target is not within the weapon's range when the 'Shoot' button
is pressed, a Toast message will appear stating the current distance and that
the shot was not performed. Otherwise, a Toast message stating the distance and
damage will be shown and a cooldown countdown will start at the appropriate
entry of the weapon selection Spinner object. Once a player has lost all health
points, a dialog appears saying that the game is over for this player and gives
him the sole option to exit the game in progress and return to the 'Main Menu'
screen. Once a player exits the game in progress, his marker will disappear on
the maps of the other players. There are no start and end conditions
implemented.

\subsubsection{The UI}

The design of the UI has been done progressively, based on intuition, one-man
tests, occasional random feedback from friends of the author and feedback from
players during the test phases.\newline

During the first phase of development two UI prototypes have been created: one
for the game screen and the other for the menus. A rough idea on what the game
will look like and what functionality has been created once the two UI
prototypes have been brought to light.\newline

We will first discuss the game screen prototype. The first challenge has been
more to create the Overlay object, get markers of three color types (red: enemy,
green: friend, blue: current player, yellow: selected player) on the map at
random positions and create the first 'Shoot' action seem real. This meant
adding a 'Shoot' button and a Spinner from which weapons could be chosen and
adding player selection into account. It had to be realistic, so a list with
Weapon objects had to be created. For this, attributes had to be conceived for
these weapons. What attributes are needed in a Weapon object? The quick answer
has been: \"damage and range\". Once this has been done, drawing the range of
the selected weapon on the map became the focus. Once that was done, shooting a
weapon had to feel like it actually happened somehow. That's when Toast
notifications have been added. After this, the realization came that those
weapons can be shot continuously. They weren't supposed to. The 'cooldown'
attribute has been added and a countdown was added on the Spinner dialog items,
once they were used. On each weapon selection if a player was selected, he would
be deselected. The range would be redrawn and the player would have to select
the target on the screen and shoot by pressing the 'Shoot' button. The UI
prototype ended up looking like Figures \ref{fig:UIPrototype1},
\ref{fig:UIPrototype2}, \ref{fig:UIPrototype3}, \ref{fig:UIPrototype4},
\ref{fig:UIPrototype5}.\newline

The following prototype would describe the basic functionality of the menus up
until the point of entering the game.\newline


\begin{figure}
\includegraphics[height=3.5in,width=6.23in]{./images/android_screenshots/ui_prototype/UI_prototype_1.png}  
\caption{\small \sl game screen prototype \label{fig:UIPrototype1}}
\end{figure}

\begin{figure}
\includegraphics[height=3.5in,width=6.23in]{./images/android_screenshots/ui_prototype/UI_prototype_2.png}  
\caption{\small \sl game screen prototype \label{fig:UIPrototype2}}
\end{figure}

\begin{figure}
\includegraphics[height=3.5in,width=6.23in]{./images/android_screenshots/ui_prototype/UI_prototype_3.png}  
\caption{\small \sl game screen prototype \label{fig:UIPrototype3}}
\end{figure}

\begin{figure}
\includegraphics[height=3.5in,width=6.23in]{./images/android_screenshots/ui_prototype/UI_prototype_4.png}  
\caption{\small \sl game screen prototype \label{fig:UIPrototype4}}
\end{figure}

\begin{figure}
\includegraphics[height=3.5in,width=6.23in]{./images/android_screenshots/ui_prototype/UI_prototype_5.png}  
\caption{\small \sl game screen prototype \label{fig:UIPrototype5}}
\end{figure}


The game screen prototype has been followed by the menu UI prototype. For
starters, this one hasn't been made with easy user interaction in mind, but
rather as a bridge for the author to think of all the necessary functionality
that should be present in everything that comes before the gameplay. Also, it
has been an exercise for the things to come. The items deemed vital have been
the main menu and a lobby screen. The main menu screen has been initially seen
as only a gate to the game, an easy intro rather than anything else. The lobby
screen has been designed to functionally accomodate what was deemed necessary in
terms of game and character preparation - and predicted the need for another
screen - that of the character settings, marked by the presence of the button
labeled 'Change my Info'. The result can be seen in Figures
\ref{fig:menuPrototype1} and \ref{fig:menuPrototype2}.

\begin{figure}
\includegraphics[height=3.5in,width=6.23in]{./images/android_screenshots/menu_prototype/MENU_prototype_1.png}  
\caption{\small \sl main menu screen prototype \label{fig:menuPrototype1}}
\end{figure}

\begin{figure}
\includegraphics[height=3.5in,width=6.23in]{./images/android_screenshots/menu_prototype/MENU_prototype_2.png}  
\caption{\small \sl lobby screen prototype \label{fig:menuPrototype2}}
\end{figure}

The two prototypes have been put together in one project that has served as the
foundation of the application to come.\newline

Very soon the necessity for a settings screen right out of the main menu came up
(The server was initially run locally, on the author's laptop. Out of all the
reasons to create a settings fragment, the first one has been that the internet
connection in the student dorm is done via VPN and thus the IP address was
changing with reconnection) and a button labeled 'Settings' was added to the
main menu. The settings fragment came right after with just an 'OK' button and
two textboxes for the IP and port inputs(as can be seen in Figure
\ref{fig:main_menu_settings}).\newline

\begin{figure}
\includegraphics[height=3.5in,width=6.23in]{./images/android_screenshots/tutorial_main_settings.png}  
\caption{\small \sl The settings screen, out of the main
menu\label{fig:main_menu_settings}}
\end{figure}

At the beginning of the development, as functionality to connect to the server
has just been implemented, there was no loading screen. Instead, the application
UI would just freeze for a few thousand milliseconds. That has been considered
unacceptable. The first idea was to add a loading widget on top the main menu
screen - but that, based on the intuition of the author, was quite unpleasant to
the eye. The loading screen had to be something separate, to mark the transition
to entering the game. A loading screen was added, with a simple loading widget
looping infinitely and a piece of 'Loading' text at the bottom of the
screen(Figure \ref{fig:loading}). This screen has remained visually untouched
until the last version of the application.\newline

\begin{figure}
\includegraphics[height=3.5in,width=6.23in]{./images/android_screenshots/tutorial_loading.png}
\caption{\small \sl The loading screen\label{fig:loading}}
\end{figure}

Once the game client became capable of successfully connecting to the server and
setting up everything for the lobby screen, the necessity for the lobby settings
screen came up - and therefore the lobby settings fragment has been created -
with a spinner for choosing between characters, a textbox for the nickname, a
multiline textbox for loading the description of each character or 'profession'
and an 'OK' button. This screen has also remained largely unchanged until the
current point of development and can be observed in Figures
\ref{fig:lobby_settings_1} and \ref{fig:lobby_settings_2}.

\begin{figure}
\includegraphics[height=3.5in,width=6.23in]{./images/android_screenshots/first_development/game_first_development_9.png}
\caption{\small \sl The character editing screen (a.k.a. the lobby settings
screen) menu\label{fig:lobby_settings_1}}
\end{figure}

\begin{figure}
\includegraphics[height=3.5in,width=6.23in]{./images/android_screenshots/first_development/game_first_development_10.png}
\caption{\small \sl The character editing screen (a.k.a. the lobby settings
screen) with the character selection dialog opened
menu\label{fig:lobby_settings_2}}
\end{figure}


\subsection{The first testing phase}

Contrary to the original plans, the first testing phase took only half a day
with another half day of preparations. Prior tests have been done by the
author alone, with two phones.\newline

A few friends(three male and two female) of the author were invited to play the
game. Three of them had Android phones. The other two have used
the two phones that the author had in posession at the time. Unfortunately, most
of them had Android 2.2 and 2.3 installed. The preparations meant convincing
them to volunteer their phones for an OS upgrade. Android 4.0 was
installed.\newline

The tests have been done mostly indoors, as they revealed various bugs in the
server and client software that were not detected when using only two phones
that were physically connected to the development computer via USB
cables. Because six people had five phones, two of them have played
alternatively. Because not all players had 3G connections available, the game
has been played by connecting to the author's WiFi router. The first bug we
noticed was that occasionally the game would disconnect for no apparent reason.
Others have been related to GPS devices in one of the phones not retrieving the
position and crashing the application or concurrency issues improperly treated.
A few temporary quick fixes got the game on track and we managed to play the
game, firstly indoors for a few times and then outside, after the author's WiFi
router has been placed on the outside window sill. This has inadvertently been a
test of the WiFi coverage of an average old 802.11 b/g router - around 50 - 70
meters radius in an open space describing a half-circle around the router. The
author was the first to accidentally run outside the WiFi coverage and get
disconneted. After the playing was finished, the author(with help from one of
the friends) took notes while everybody has expressed their criticism of the
gameplay, UI and game satisfaction. Although the concept itself has been
positively appreciated, the game UI has received a lot of criticism. Also the
random disconnection of the game from the server has caused a lot of frustration
among the test players. A proposal has been made to allow saving the last
player status and position on the server and allow a timeout until the player
would not be allowed to reconnect. A debate led to the conclusion that
enabling such a feature could allow cheating, through the following scenario: A
player would disconnect from the game, get close to other players, reconnect,
shoot and repeat the sequence - thus avoiding damage and causing frustration to
others. \newline

Until the end of the first testing session done with a group of people, the UI
has remained largely untouched. The two unused buttons on the game screen have
been kept for getting the players' feedback on how pleasant the position of
those buttons is for the reach of their fingers. The goal for the game UI has
been to become similar to a game controller, but with the game screen in the
middle. And the screen had to be layed out in a way in which the user would not
feel cramped when playing the game. The only addition to the UI has been
the 'Settings' button in the main menu, together with the settings
screen.\newline

The first people to test the game have criticized the following aspects:
\begin{enumerate}
  \item \textbf{The game UI}: The need to reselect the target every time the
  weapon is switched has proven frustrating. The entire process of selecting the
  weapon and only then firing it has been regarded as too slow: the players
  wanted to be able to shoot the entire arsenal immediately, if
  possible.\newline
  
  Another issue has been that the 'Shoot' button was disproportionally small
  compared to the weapon selection spinner, in the testing phase. After showing
  the players the original test UI, they still deemed the 'Shoot' button too
  small.\newline
  
  The lack of control over what is happening has been criticized: The player
  health was only displayed above the head of the player, along with the name
  and 'profession'. It made it hard for players to notice changes. Notifications
  on who 'shot' who were not yet implemented. The player's health was not
  displayed anywhere. Messages could not be sent to the team. The first proposal
  has been to enable writing and sending messages, SMS-style. After some
  discussions, the conclusion came that predefined messages are much more
  useful(like, say the ones used in the video game \"CounterStrike\"), while
  'custom' messages can be communicated verbally, keeping the players more
  focused on the dynamics, rather than on the technicalities of the game.
  
  \item \textbf{The menu UI}: The lobby UI has been criticized - but not for
  it's layout, but for the comprehensibility - there was no mark for the
  player's 'Ready' status, the 'Change my Info' label on the button that leads
  to the 'profession' and nickname editing screen wasn't comprehensible. The
  nickname and 'profession' were not remembered from one game session to the
  other or from one app launch to the other. This frustrated the users - and
  they simply refused entering their nicknames all the time. The game ended up
  with no one knowing who's who, because they had the same nickname - the
  default 'Player'. One tester complained that he wanted to click his own name
  in the team lists and get to the settings screen.\newline
  
  A general complaint has been expressed towards the fact that there are no
  tutorials on how to play the game.
\end{enumerate}

The first testing phase, though short, has left a lot of guidelines on what to
do from that point on.\newline

\subsection{The second develpment phase}

The first testing phase has left a lot of 'todos', notes and guidelines for
better adapting the UI and game mechanics to the player's needs. The second
development phase has addressed these needs with a personal touch from the
author and an influence of the switch to a newer version of Maps API.

\subsubsection{The Server}

The only change done on the server in the second development phase, beyond bug
solving, has been the switch from InetAddress to InetSocketAddress as key in the
server's management HashMaps. That's because multiple connections from behind a
router with NAT were not possible. InetAddress only holds the remote IP of the
connection. The InetSocketAddress now holds both the IP and port of the
connection.

\subsubsection{The Client}

The client is an Android application that uses one Activity and multiple
Fragments. 

The client uses seven fragments for the UI: 'Main Menu', 'Info and
Tutorials', 'Settings', 'Lobby','Lobby Settings', 'Loading' and 'Game':

\begin{enumerate}
  \item 'Main Menu' : It has been slightly modified: The 'Start Gane' text has
  been changed into 'Connect to Server' - making its functionality more
  obvious. The logo presented in Figure \ref{fig:game_logo} has been added for
  user feedback. The 'Info and Tutorials' button has been added - it leads to
  a new Fragment that presents the idea adn functionality of the game.
  
  \item 'Info and Tutorials' is a fragment with a number of buttons and a
  ScrollView for displaying text and images. Here, the user will find
  instructions for the purpose how use of the app and detailed descriptions of
  the functionality of the 'Lobby' and 'Game' UIs.
  
  \item 'Settings' has not been modified.
  
  \item 'Loading' has not been modified.
  
  \item 'Lobby' : it has been slightly modified to remember the last used
  nickname and 'profession'. Each player in the team lists now shows the ready
  status of that player and a '\[ME\]' indicator has been added to the current
  player so that he can identify himself in the list. Clicking on one's own
  nickname in the list now navigates to the 'Lobby Settings' screen for
  nickname and 'profession' change. Two background images have been tried out on
  this fragment, for user feedback. They are shown in Figure
  \ref{fig:game_background_black} and Figure \ref{fig:game_background_white} 
  
  \item 'Lobby Settings' has not been modified. A background image has been
  tried out as background, for user feedback. See Figure
  \ref{fig:game_background_lobby_settings}
  
  \item 'Game' has been completely modified. Due to changes in the use of the
  Maps API and tester feedback, it has gone through the most changes. At this
  point, the game screen looks as follows: for all the 'weapons' and 'powerups',
  buttons are aligned starting from the bottom left corner of the screen and
  extending them to the right. From the bottom right corner and going up, on the
  vertical axis, three buttons and a TextView exist: the 'Previous Target',
  friend/foe toggle, 'Next Target' buttons and a TextView that shows the
  distance to the next selected player. If no player is selected, the presented
  text will be '0(Self)'. Otherwise, just the distance measurement is shown,
  with no text. In the top right corner, the default 'My Position' button is
  shown - with its specific icon. In the top left corner of the screen, a
  TextView with large text shows the remaining 'health points' of the player.
  Underneath this TextView lays a spinner for sending predefined messages to the
  team. The full functionality of the UI will be presented below.
  
\end{enumerate}

The typical app use scenario goes as follows: The player enters the game and
sees the 'Main Menu'. The app checks if Google Play Services are installed and
if the GPS is turned on. If either of these conditions is not met, a dialog is
presented: the player must choose to install Google Play Services and/or turn on
the GPS or exit the game. Each dialog directs the user to the appropriate Google
Play or Settings page - where the player only has to either click 'Install' or
switch the 'Enable GPS' toggle to 'ON'.\newline

Once there are no more requirements to be met in order to play the game, the
player can click on 'Connect to Server', be briefly presented with the 'Loading'
fragment and then enters the 'Lobby'. There, if the app was started for the
first time, he can see the default nickname 'Player' and the default
'profession' - Marine. If the app was previously used, the player will see the
last nickname and 'profession' used in previous runs. The server distributes the
players according to team sizes, so the player has an equal chance to see
oneself in either the 'Home' or 'Away' team. Then, the player can opt to change
the team by using the 'Change Team' button. Also, the nickname and 'profession'
can be changed by navigating to the 'Lobby Settings' screen. Once the 'Ok'
button is pressed in the 'Lobby Settings' screen, the changes are saved in the
app and sent to the server for broadcasting, and the player is returned
to the 'Lobby' screen.\newline

At the point where all the players have done setting up, they can send the
'Ready' signal. When all the players are ready (for testing purposes, one
player present on the server is enough to start a game), the five second
countdown is received from the server and the player is now facing the 'Game'
screen.\newline

In order to understand what can be done at this point, a detailed description of
the 'Game' UI is necessary: \newline
 
\begin{figure}
\includegraphics[height=3.5in,width=6.23in]{./images/android_screenshots/tutorial_game.png}  
\caption{\small \sl The in-game UI \label{fig:game_ui}}
\end{figure}

We can visualize the UI based on the screenshot presented in Figure
\ref{fig:game_ui}. It presents the groups of UI elements on the screen.
We shall now present all of them, separated into groups, by position
(positioning also separates functionality groups, therefore we can also state
that they are presented by related functionality). All the UI elements on the
'Game' UI are programmatically generated - and not from an XML file  :
\begin{enumerate}
  \item\textbf{The bottom of the screen, starting from the bottom-left corner}:
  The 'weapon'/'powerup' buttons are generated once the 'Game' fragment is loaded,
  based on the list of weapons specific to the player's 'profession'. They
  appear from left to right. If less or equal to three weapons are given, the
  buttons will be made slightly larger than otherwise - up to a maximum of six
  (the number of weapons featured in the test 'profession', 'All Encompassing').
  Each button press shoots the 'weapon' or 'powerup', according to a so-called
  'policy' provided in the Weapon object. The policies are as follows: 'self',
  'friends', 'friends and self', 'enemies', 'friends and enemies', 'friends and
  enemies and self'- stating on which kinds of players one given 'weapon' or
  'powerup' can be used. A button long press will draw the range of the
  'weapon'/'powerup' on the map, with the player's position in the center.
  
  \item\textbf{The bottom-right corner of the screen}: The player selection
  buttons and the distance indicator are generated on the vertical axis, starting from the
  bottom-right corner of the screen - in order, the 'previous target',
  friends/enemies toggle, 'next target' buttons, and on top a TextView showing
  the distance to the selected player or '0(Self)' if no player is selected(in
  translation, the 'self' or 'current player' is selected). The friends/enemies
  toggle is used for the obvious reason of quickly choosing between one of the
  two types of players besides the 'self'. The 'next target' button's
  functionality is to find the next closest player from the selected category.
  Let's say that the closest player is already selected. Then, on a press of the
  'next target' button, the second closest player will be selected and so on.
  The reverse applies to the 'previous target' button - which selects the
  previous farthest player if a player is selected and the closest one
  otherwise. The distance indicator serves for the use of an experienced player
  who knows the weapon's distances by heart and does not want to the lengthy
  long click operation to get the weapon's distance shown on the map. The
  players may also be selected by simply tapping their respective markers on the
  map - but through testing it has been determined that in a more intense and
  dynamic situation, selection by clicking on the marker can become difficult
  and inaccurate. Both modes are supported now. One other aspect of player
  selection is the now-vital click in a random unoccupied area on the map for
  deselecting all players(and selecting oneself for, say, self-healing or
  activating powerups on oneself).
  
  \item\textbf{The top-right corner of the screen}: The 'my position' button
  that is provided as an option through the Maps API. Clicking it will cause an
  animation to scroll through the map until the position of the player is
  centered.
  
  \item\textbf{The top-left corner of the screen}: The health indicator is a
  TextView that presents in large text the remaining health points of the
  player. Underneath it lies a spinner with a list of messages that the player
  can send to his/her team. One can do so by clicking on the UI element
  representing the spinner, beneath the health indicator. After the first
  click, the spinner dialog appears and shows the list of messages. The
  player can send one of the messages in the list by clicking it. The dialog is
  canceled by clicking outside it.
  
  \item\textbf{The area in between the two top corners of the screen}: This is
  the area where the powerup duration is dynamically shown during its effect.
  This can be seen when the player casts 'invisibility' or 'shield' on oneself
  or when a friend casts 'shield' on the player
  
  \item\textbf{The area above the weapon buttons}: This is where the
  notifications appear, in the form of Toasts. By default, a toast has a given
  lifespan - but each time a new toast needs to be displayed, it first closes
  the toast currently on display, if it is the case.  
\end{enumerate}

Although they are not yet discretized, thus far we can distinguish three types
of game that can be played with the current setup: the \textbf{default
game}(which requires multiple players, but excludes the 'All Encompassing' game
type), \textbf{duel}(two players choose the 'All Encompassing' profession -
receiving the entire arsenal provided by the game - and try to take each other
out by using various strategies of combining 'weapons' and 'powerups'),
\textbf{David vs. Goliath}(a small number of players choose the 'All
Encompassing' profession and play against a significantly larger number of
players that use all the professions except 'All Encompassing').\newline

Until now, the default game and the duel have been tried out - and they are
fit for different situations: The default game can take up to 20 minutes and is
played across distances varying from small to very large, depending on the mood
and disposition for running of the players. The duel is performed usually by two
players sitting next to each other and takes two-three minutes.\newline

Thus far, no game end condition has been introduced - players who lose all
health points are notified that the game is over for them through a dialog that
gives them two options: quit the game or spectate. The second option implies
that they are marked as 'dead' on the map, cannot be selected and their weapon
buttons are removed from the screen. Still, they can watch the evolution of the
game on their mobile devices. When all players of one team are marked as 'dead'
(they have lost all their 'health points') and at least one player of the other
team still has more than zero health points, it can be considered a win for the
other team. The ending condition has not been implemented, because for this
stage of development of the app it would be just a dialog stating the obvious
'You have won !' message. A more interesting feature may be added in the future,
if imagination or user feedback provides it - or the game may be integrated in a
larger context.\newline

The main functional differences from the first development of the game are:

\begin{enumerate}
  \item A switch from Google Maps API V1 to V2 has been done. This implied
  rewriting the whole UI and several methods in the helper libraries.
  
  \item Because of the switch mentioned above, the custom Overlay object (along
  with the custom tap and marker drawing functionalities), the object
  responsible with position retrieval have been discarded. Position retrieval is
  done through an option provided by the GoogleMap object provided by the Google
  Maps V2 API. The whole marker and weapon range drawing are also done through
  the GoogleMap object - OverlayItem objects have been replaced by Marker
  objects and GeoPoints have been replaced by LatLng instances.
  
  \item The whole UI has been rewritten, having as a result what was described
  above.  
  
\end{enumerate}

Passive disconnection detection functionality has been added in the loop that
waits for incoming messages. The functionality of some weapons and powerups has
been modified. The `Radar` has been removed. The 'Invisibility cloak' and
'Knife' have been added. Powerup duration indicators have been added.\newline

The final message structure in between the server and the client will be
presented in one of the Annexes.\newline

During the second development phase, a lot of decentralization and
modularization of functionality has been done. Classes that only served the
Game UI relying on Maps API V1 have been removed. Others, serving the Maps API
V2 have been added. Various bugs that caused crashes have been corrected. The
starting condition has been reintegrated in the game - it disabled the weapon
buttons until the 'Start Game' message would be received.\newline

\subsubsection{Logging, Testing and Data Usage}
After the first testing phase, the need to log app crashes has come up - as
sometimes it can prove quite difficult to recreate the crash conditions while
solo testing. A logging system based on Logcat capture has been added. Uncaught
exceptions are now caught and logged(they are always the ones crashing the app)
before the app closes. One issue currently not dealt with and considered
benign for the prototype is that the log files(created through classical Java
functionality on the SD Card) can only be seen from the mobile device and not
from a computer communicating with it via USB. This issue was not considered
stringent, as the number of testers and testing devices is still quite low and
manual extraction of crashes is still easier than implementing a solution(The
problem is that these files are signed with a UID and thus cannot be seen by
external devices).\newline

A lot of testing has been done with the 'UI/Application Exerciser Monkey' to
show up potential bugs and crashes. Some bugs have been found and patched - but
one has persisted: Apparently, there are some humanly-impossible combinations of
keyboard and touch inputs that lead to a Spinner dialog crashing the
application. \newline

Another bug that has persisted is that of the WiFi disconnection(3G connections
stay alive). Apparently, this is an Android bug.\newline

As for logging a Logger class has been created, its functionality has been
extended to retrieve the data usage of the app. Because of various compatibility
reasons, the data usage of the entire device during the usage of the app is
retrieved, instead of the usage of the app itself. This means that the data
usage of, say, social network or mail applications adds to the count - but as a
general idea (and not very precise numbers) is needed, it has been considered a
decent solution.

\subsubsection{The UI}

The second development session has come with the mindset to completely change
the UI. And not only that was done: the whole underlying framework has been
changed. A switch to the Google Maps V2 API has been made. A lot of
functionality has been thrown away - such as the Overlay that served for drawing
the players and for laying out the controls. A lot of new controls had to be
added. A lot of underlying technicalities have been changed(such as the custom
screen tap functionality and objects used for positioning).\newline

Because of the switch to the new Maps API, the MapView object which served as
the container for all controls has disappeared. It has been replaced by a
GoogleMap object which is not intrinsically a View and cannot be treated thusly.
Some improvisation was required to artificially create a container. This has
been done by programmatically creating a RelativeLayout and adding the View that
contained the map as a child. Then, the apparent choices have been to either add
all the controls programmatically - which is very tedious - or embed a Fragment
within another Fragment - which is downright wrong. All the controls have been
generated and added programmatically. And the result has been previously
described in the paper - having as a result the UI presented in Figure
\ref{fig:game_ui2}: The UI has been properly partitioned in four functional
areas that, although they provide much more control and functionality, don't
cloister the map. The health of the current player is visible, a button has been
added to center the image on the current position of the player, individual
buttons are generated for the list of weapons provided for a 'profession', a
group of buttons is added for help in target selection. Target selection by
tapping on the screen has been preserved, but now is enhanced through the 'Next
Target' and 'Previous Target' buttons. The latest addition - the Spinner object
serving message sending was taking up too much space and hindered the view of
the map. A good solution has been to make it semi-transparent - thus, making it
non-intrusive, yet present. The ranges of the weapons are now drawn in green on
demand, by long pressing the weapon buttons. The blue circle around the player's
position is the default indication of the GPS accuracy. Thus far, it has not
been removed. \newline


\begin{figure}
\includegraphics[height=3.5in,width=6.23in]{./images/android_screenshots/second_development/game_second_development_5.png}
\caption{\small \sl The game UI \label{fig:game_ui2}}
\end{figure}

The menu UI has also been modified:
\begin{enumerate}
  \item \textbf{The main menu screen}: The welcome message has been removed and
  place has been left for a potential logo. A logo has been tried, as seen in
  Figure\ref{fig:logo_white}. Also, a button to navigate to the newly added info
  and tutorials screen has been added.
  
  \begin{figure}
  \includegraphics[height=3.5in,width=6.23in]{./images/android_screenshots/logo_white.png}
  \caption{\small \sl A white game logo \label{fig:logo_white}}
  \end{figure}
  
  \item \textbf{The info and tutorials screen}: A new Fragment has been added.
  It contains four buttons that lead to information and tutorials on what the
  game is and how to play it. It also features a 'Back' button. The initial plan
  was to have a WebView as container and load the tutorial articles in HTML
  format. Then, the author subjectively deemed HTML creation too complicated for
  the task and the decision has been taken to have a ScrollView as container
  instead and use a LayoutInflater object to generate the UI from XML files. The
  results can be seen in Figures \ref{fig:tutorial_fragment1} and
  \ref{fig:tutorial_fragment2}.
  
  \begin{figure}
  \includegraphics[height=3.5in,width=6.23in]{./images/android_screenshots/tutorial_fragment_1.png}
  \caption{\small \sl Tutorials screen: the About
  page\label{fig:tutorial_fragment1}}
  \end{figure}

  \begin{figure}
  \includegraphics[height=3.5in,width=6.23in]{./images/android_screenshots/tutorial_fragment_2.png}
  \caption{\small \sl Tutorials screen: the Game Guide page
  \label{fig:tutorial_fragment2}}
  \end{figure}
  
  
  \item \textbf{The lobby screen}: The lobby screen has also been modified: The
  requested feature to map the click on one's name to the character settings
  screen has been added. Also, the \"ME\" marker has been added to distinguish
  the current player from the rest and the \"READY\" marker of one's 'Ready'
  status. Two backgrounds have been tried on this screen, as seen in Figures
  \ref{fig:lobby_background_black} and \ref{fig:lobby_background_white}.
  
  \begin{figure}
  \includegraphics[height=3.5in,width=6.23in]{./images/android_screenshots/second_development/game_second_development_3.png}
  \caption{\small \sl Lobby screen with black background
  \label{fig:lobby_background_black}}
  \end{figure}

  \begin{figure}
  \includegraphics[height=3.5in,width=6.23in]{./images/android_screenshots/lobby_background_white.png}
  \caption{\small \sl Lobby screen with white background
  \label{fig:lobby_background_white}}
  \end{figure}  
  
  \item \textbf{The lobby settings screen}: The lobby settings screen has been
  kept almost intact since its creation. The functionality to remember the last
  used player's nickname and 'profession' has been added. Also, two backgrounds
  in black and white have been tried out. They can be seen in Figures
  \ref{fig:lobby_settings_background_black} and
  \ref{fig:lobby_settings_background_white}.
  
  \begin{figure}
  \includegraphics[height=3.5in,width=6.23in]{./images/android_screenshots/second_development/game_second_development_8.png}
  \caption{\small \sl Lobby settings screen with black background
  \label{fig:lobby_settings_background_black}}
  \end{figure}

  \begin{figure}
  \includegraphics[height=3.5in,width=6.23in]{./images/android_screenshots/lobby_settings_background_white.png}
  \caption{\small \sl Lobby settings screen with white background
  \label{fig:lobby_settings_background_white}}
  \end{figure}
  
\end{enumerate}

\subsection{The second testing phase}

As the first testing phase, the second one has taken much less time than
initially planned - only one day was necessary. A larger number of friends of
the author were invited. More phones have been brought and almost all had 3G
connection. This time, the app supported Android versions starting with 2.3
- which made it much easier.\newline

A large part of the testing phase has been the preparation: installing the
application on all devices, connecting everyone to the wireless network and VPN
(At the moment of testing, the author lived in a student dorm - and Internet
access was provided through a secure VPN connection. But because everyone's 3G
plan included at most 500MB, Wifi was the first choice). This has proven to be
very frustrating - therefore, everybody switched to 3G and started to play -
firstly indoors and soon after, outdoor. The UI changes have been positively
appreciated by everyone - yet one feature has proven to be quite useless: the
long click on the weapon buttons. The long click draws the range of the weapon
on the map. Our testing scenario, though, has not taken place over large
distances(no one went out of Sniper Rifle range - 150 meters) - therefore the
gameplay has been very fast paced (a game lasted on average between 10-15
minutes) and no-one has actually used the functionality. The distance indicator
has also not been noticed by most. The health indicator, target selection
buttons and shooting functionality of the weapon buttons have been appreciated
as 'right', 'appropriate', 'useful'. A request was made for a finishing
condition and a 'You have won' screen for the winner team - and some sort of
reward system. Another feature that has proven to be more of a liability in this
situation was the starting condition. In the first few games(played indoors),
the players 

Unlike before the first testing phase, when few to no people interacted with
the application, before the second testing phase a large number of the friends
and acquaintances of the author have seen the application and have been giving
constant input as to whether certain aspects of the game are worthwhile or not.
Also this has been when discussions on future development of the game have
started. A lot of conversations have been recorded to audio files and notes on
paper. Among the discussions, the ones on the current development have been
focused on visual aspects: The background images, the logo; a proposal to change
all the markers with 'profession'-specific ones and to replace the text on the
buttons with symbols. Besides all this, a discussion has started on the name of
the application itself. Both very positive and very negative opinions have been
expressed, but none of indifference. The first name chosen by the author for the
game has been \"People with Guns\" - but after debate and player
feedback it has been later changed to \"Gun Run\".\newline

There were ten people who actively participated in the discussions and debates
over the aforementioned aspects of the game: five male and five female.\newline

The logo has been strongly rejected by everybody - it was deemed too colourful
and unrelated to the significance of the game - a semi-casual game to promote
enjoyment and social bonding.\newline

The black background images used in the lobby and lobby settings screens have
been positively appreciated by the males but received mixed opinions from the
females: Three of them said that they inspire violence - which is unrelated
to the game itself, while the other two have deemed them appropriate.\newline

After being presented with the white background images for the lobby and lobby
settings screens, everybody has rejected them, uninanimously stating that they
break the atmosphere of the game and that it should promote a level of
competition. Everybody concluded that they prefer darker colors in the menus and
the contour of the map in the background is one good step towards something that
defines the game.\newline

The second testing phase took a bit longer than the first. More players tried it
out, more games have been played and - most importantly - more games have been
played outside.\newline

The conclusion of the second test phase has been that the tutorials are too
long, too descriptive. The menus haven't received any criticism - but the game
UI has: The long click functionality to draw the ranges of the weapons has not
been used at all. The distance indicator was not even seen by the players and
was therefore not used in any way. The 'Invisibility Cloak' powerup is used only
by the current player on self, but in the current implementation it requires
selecting oneself(deselecting everyone else) - this was deemed frustrating. The
entire concept of selecting oneself by deselecting all others(tapping an area
on the map where no markers are present) was deemed frustrating. Everybody
playing(just two of the people from the discussions before the second testing
phase have been present) has expressed the necessity of symbols instead of text
on the buttons and on the map. Another aspect of criticism has been the order of
the 'weapon'/'powerup' buttons: they should be separated into two groups:
'weapons' and 'powerups'.\newline
