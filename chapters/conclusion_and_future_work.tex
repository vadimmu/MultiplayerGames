
\section{Future work}


This project has served as a proof of concept, a demonstration for the
playability of a GPS-based Real Time Tactics game with some Shooter elements. As
it has proven enjoyable and easy to play for everybody who tried it.\newline

During the development and testing of this application, requests and ideas have
come up for future versions of the game. They will be presented in
order:\newline
\begin{enumerate}
  \item Icons will replace the names of all weapons and the professions of the
  players. A logo will be added and the menus will have suitable backgrounds.
  The current backgrounds are to be replaced.
  
  \item A further step in the development of the game would be to add RPG
  elements to it: Mechanics will be modified to add elements of realism to the
  game. Shotguns will be most effective in close range, rifles at longer range.
  Damage will be calculated based on range versus ideal range. 'Missing' a shot
  will become a factor in the game. Skills to use some weapons more effectively
  or to receive less damage will be subject of improvement. Health will also be
  subject of improvement. Two play modes will be added: casual and professional.
  In casual mode, character level will be irrelevant - thus allowing people to
  play, regardless of their daily implication in the game. In professional mode,
  players will be able to take the game to a further level and become
  competitive.
  
  \item Another aspect to the game will be adding an incentive for travel and
  discovery. Each player will be able to get temporary or permanent items by
  exploring the sorroundings of his neighborhood or travelling to remote
  destinations.   
  
  \item An MMO game will be developed, having this game and its variants as
  subgames. The concept is as follows: An application will be linked to a social
  network, such as Facebook. Players will be able to log in to the application
  with their Facebook accounts. They will set up who can see them on the map and
  who they want to see themselves: only themselves, friends, friends of friends.
  They will be able to interact with each other directly through the app and
  invite each other to play games, such as the one developed here. Details such
  as creating custom avatars will be provided through the application or a
  website. A scoring system will be added and competitions will be held through
  the app - they will be advertised through Facebook events. Joining the event
  will ensure participation in the competition. Facebook statuses might be
  published by the app, if the player wants so: entering, winning or losing a
  game, retrieving items, various accomplishments can be subject of publishing
  as Facebook status updates.
  
  \item Quests will be ultimately added to various locations, if it is deemed
  reasonable. This is still subject to debate, as the quests will be localized
  and the game experience will then vary greatly from country to country.  
  
\end{enumerate}

Other various technical modifications will be made to the game: In the first
improvement iteration, Websockets will be implemented as communication protocol.
Also, a server that would host a larger number of games is vital (now the
server can only host one game).


\section{Conclusion}

The purpose of this project has been to develop a socially engaging mobile game
to a broad spectrum of population. In translation, this means that the purpose
of this game is to be entertaining and enjoyable to people of both sexes and all
ages, with or without technical skills or gaming background. \newline

As the field testing has proven, the goal has been achieved: All the game
testers have enjoyed it: roughly fifty percent female - fifty percent male
population, ranging from the age of 18 up to 27 have enjoyed playing and
requested the opportunity to play again. Because of the small scale of the
tests though, a larger age palette could not be tried out within the setting of
this project.\newline

With \"People with Guns\" a new niche has been created in the genre of
location-based games. That is of location-based Real Time Strategy games - and
location-based Real Time Tactics games.\newline

For me, the author, this has been a journey into how an idea is mined for and
developed, step by step, try after try, using scientific reasoning
alongside trial and error. Through this project, I have learned Android
programming, applied Agile methodologies, devised a personal reasoning for the
cases in which Unit Testing is applicable. I have made server prototypes with
Node.JS, Python, Java, Map prototypes with the Google Maps V1 and V2 APIs, tried
out Websockets implementations for Node.JS, Python and Android, tested the usage
of both Google GSON and Jackson libraries for JSON serialization and
deserialization. I have evolved an idea into a mature application with optimized
data and hardware resource usage, with its own messaging and configuration
system. The end result is a fast-paced game with a dynamic UI that has been
optimized for ease of use and a set of 'weapons' and 'characters' that fit
various types of player personalities. From now on, development will rely on a
well set foundation.\newline


Concomitant with the scientific approach of the conceptualization and
development of this mobile game subgenre, this has been a very subjective,
personal experience. A lot of intuition and communication have played a large
part of what is now the real time tactics game prototype \"People With Guns\" .
A lot of choices have been made on feeling and personal perception. The people's
opinions on the application, design aspects and even the name of the game itself
have been put to question and have received very varied feedback. I have seen
hands-on what changing a background image from black to white can cause in the
player's attitudes. I have seen difference of opinion and perspective between
sexes and ages. This research has transcended the defined task of developing a
game type that has not been tried before. It has become a very personal
experience and came with its hardships and teachings. Although the entire paper
is expressed scientifically, I have tried to write it in a manner as personal as
possible.\newline

The usage of 'he' should be replaced by 'she/he' or 'he/she' and has only been
used this way for the convenience of the author and no other reasons.
