\section{Related work}

In this chapter, we will follow the games and concepts that have lead to the
assembly of the idea for the game developed on this project. We will define a
number of computer game genres, classify some of the most popular location-aware
games, game platforms and non-digital games that aided the concept and present
the evolution of the idea for this game prototype. The game concept brought up
here is part of a genre that has a long legacy among computer games, but is not
yet known to the location-aware mobile context. The genre is that of `Real
Time Strategy`(RTS)\cite{rts} games. Some other elements come from the genre of
'Shooter' games\cite{shooter}. We will do that with a subgenre of RTS, called
Real Time Tactics(RTT)\cite{rttvsrts}. For reasons that will be presented
throughout this paper, the type of game can be considered an RTT / Shooter
hybrid. \newline


\subsection{Introduction: Real Time Strategy games}

We will first make the introduction of Real Time Strategy and its subgenre, Turn
Based Strategy

\subsubsection{What is Real Time Strategy?}

RTS games are defined as real-time (continuous time) competitive games, in which
several players fight against each other, either in a skirmish or team versus
team. The purpose is to defeat the opponents by taking real-time decisions on
managing resources and troops. The focus is, therefore, on three key elements:
\textbf{managing resource gathering}, \textbf{building a base to provide troops}
and \textbf{battle tactics}.\cite{rts}\newline

This project will focus on a subgenre of RTS, Real Time Tactics(RTT) - which,
instead of managing all three aspects of RTS, focuses on battle tactics.\newline

Because it is essentially a simplified approach to RTS\cite{rttvsrts}, RTT can
provide a good proof of concept and at the same time simple and enjoyable
playability. The advantage of having a RTT GPS-enabled game versus RTS is that
it greatly reduces the play time, requires less skills and has the potential of
being less stressful than RTS.\newline

\subsubsection{Real Time Strateg versus Turn Based Strategy}

Turn Based Strategy games\cite{rtsvstbs}(such as chess and board games, for
example) allow the player to take his time and plan every move. Implicitly, the
duration of the game is greater. This type of game is not in the scope of this
project, yet it deserves mentioning, as there are many such games for mobile
devices. Some notable implementations are the ones of Scotland Yard(Ravensburger
GMBH), Catan(Catan GMBH) and Monopoly(Hasbro, Inc.).\newline

In particular, there are two implementations for Scotland Yard that have been
conceptually interesting in the search. A comparison has been
made\cite{rttvsrts2} on whether it is better to port a board game in a
real-time(continuous-time) or turn-based mobile GPS-based game. These two
implementations are Mobilis XHunt(turn based) and MisterX Mobile(real-time). The
comparison has been made on 10 aspects: Fun, Smooth Progression, Dynamic
Gameplay, Ease of play, Stressless Gameplay, Communication, Strategy, Clear
Rules, Low Risk, Education. The conclusion was that there was no favorite
between the two, but it has been concluded that these 10 aspects have different
weights for the player and that Fun, Smooth Progression and Dynamic Gameplay
have a higher individual weight to players than the rest.\cite[p.5]{rttvsrts2}
\newline

Based on the knowledge gathered from the above-mentioned comparison, the aim of
this project will be to maximize the interactivity and dynamics of the RTS game.


\subsection{Location-Based Augmented Reality Games}

Ever since the first operating system for handheld devices, the spreading of
smartphones has intensified every year.\newline

The rapid expansion of mobile computing presents new challenges and
opportunities for both the user and the developer. The ease of use and the
presence of touchscreens, GPS receivers, gyroscopes, accelerometers and, since
recently, even barometers gives way to new approaches in developing
games.\newline

Smartphones equipped with GPS receivers are becoming ubiquitous. This makes way
to the propagation of GPS gaming. Although the concept is old, very few attempts
have been made in this direction and this branch of game development may be
considered to still be in one of its early stages of maturity.\newline

This chapter is a proposal for extending the genre of mobile real-time
multiplayer location-based games.

\subsubsection{Location-based games}
Location-based games take advantage of the mobile devices' built-in receivers
for global positioning. They provide the user's location with an accuracy
ranging from a few to several dozen meters. Because the most mobile devices in
the world today rely on the Global Positioning System(only recently support for
GLONASS has been added to smartphones), we can use the term GPS-based
games.\newline

GPS-based games came up long before this feature has become ubiquitous in mobile
phones and tablets. One of the first widespread GPS-based games is Geocaching.
It is composed of two parts: 
\begin{enumerate}
  \item Placing physical caches at various locations that can be considered
  interesting or worth visiting and publishing their GPS positions (eg. on
  websites).
  \item Searching for various caches by using a GPS device.
\end{enumerate}

Along with the evolution of smartphones came that of the mobile games. GPS games
come in a lot of flavors, from GPS-based tours, adventure and investigation
games to various race games - single and multi player and massively multiplayer
online games.


\subsubsection{Types of GPS-based games}

There is a number of GPS-based games and game authoring tools that are
available for iOS and Android devices. The ones studied for this paper are :
ARIS, Tourality, Wherigo, conTAGion, Shadow Cities, SCVNGR, Please Stay Calm,
Parallel Mafia, Parallel Kingdom, Tripventure, Warfinger, Totem, Portal Hunt,
aMazing, Ingress, MobileWar, Mister X Mobile, Mobilis XHunt, Own This World,
MapAttack.\newline

For better understanding the classification done below, we will first define
each type of game:

\begin{enumerate}
  \item \textbf{Adventure/Investigation Game} - Game in which the player plays
  the role of a character in a story. The primordial characteristics of this
  genre is that it is focused on immersion in the story, puzzle solving and
  investigation, rather than on physical skills. Also, the tendency of this
  genre is towards single player experience, though occasionally multiplayer is
  also implemented (eg. the ARIS-based game 'Mentira').
  
  \item \textbf{Massively Multiplayer Online Game} - This type of game is
  designed to support large numbers of players (even in the number of millions
  in some cases) that play and interact in a persistent virtual world. This type
  of game allows both cooperative and competitive gameplay and is exclusively
  based on multiplayer. Subgenres include MMO Role Playing Games and
  MMO Shooters.
  
  \item \textbf{Casual Game} - Analogous to the MMO, the Casual Game is targeted
  at mass at a mass audience and can incorporate any type of game type. The
  particularity of this genre is that it aims at having simple rules, simple
  gameplay and requiring no specialized skills. 
  
  \item \textbf{Racing Game} - It's a genre defining a broad range of games. In
  the case of computer games, it describes mostly motorized vehicle racing. In
  the mobile context, it mostly describes racing on foot against time or through
  a number of checkpoints.
  
  \item \textbf{Shooter Game} - This one's a subgenre of action games. It
  focuses on first or third person experience, speed, aiming and reaction time.
  Usually the weapon is ranged, although close-combat weapons are included in
  most games.
  
\end{enumerate}

We will now classify the games/platforms based on the genres they best fall in:

\begin{enumerate}
  \item \textbf{Adventure/Investigation Games}
  		\begin{enumerate}
  	  		\item ARIS
  	  		\item Wherigo
  	  		\item Tripventure
  	  		\item Tidy City	  
  		\end{enumerate}
  		
  \item \textbf{Massively Multiplayer Online Games}
  		\begin{enumerate}
  	  		\item Shadow Cities
  	  		\item Please Stay Calm
  	  		\item conTAGion
  	  		\item Parallel Mafia
  	  		\item Parallel Kingdom
  	  		\item Portal Hunt
  	  		\item Ingress
  		\end{enumerate}
  		
  \item \textbf{Casual Games}
  		\begin{enumerate}
  	  		\item SCVNGR
  	  		\item Warfinger  	  		
  	  		\item aMazing 	
  	  		\item Own This World
  	  		\item MapAttack	  	  		
  		\end{enumerate}
  		  		  		
  \item \textbf{Racing Games}
  		\begin{enumerate}
  	  		\item Tourality  	  		 	  		  	  		
  		\end{enumerate}
  		
  \item \textbf{Shooter Games}
  		\begin{enumerate}
  	  		\item MobileWar	  		  	  		
  		\end{enumerate}  
\end{enumerate}

A special category is represented by Mister X Mobile and Mobilis XHunt, which
both bring a board game (Scotland Yard) to the mobile environment. While the
former adapts the board game to real-time gameplay(placing it closer to the
'Multiplayer Racing Games', with elements from 'Real Time Strategy Games'), the
latter falls in the definition of 'Turn Based Strategy' games.

\subsubsection{Physical games}

A number of games have played a significant role in the construction of the
ideas that led to the eventual assembly of the one that was decided for
implementation. These games originate in the years before the digital age or the
widespread of computers and GPS-enabled mobile devices. Some of these games have
already been implemented in mobile games: tag and racing games and geocaching.
We will mention all of them and briefly describe them and where they came in
conceptually:
\begin{enumerate}
  \item \textbf{Orienteering} is a sport in which a map and a compass are used
  to navigate through checkpoints, across unfamiliar terrain. Checkpoints
  colored in white and orange are placed at various features along the way. This
  sport has started as a training discipline for the military and later on
  expanded as a civilian sport.
  
  \item \textbf{Radio Orienteering} is worth mentioning separately, as it was in
  itself an important source of inspiration for the games proposed along the
  way. It features a radio for direction finding, besides the map and compass. 
  
  \item \textbf{Rogaine} is a cross-country navigation sport. As a difference
  from orienteering, where a path must be followed, in rogaining a larger number
  of checkpoints is given and there is no time limit. A team of two to five
  people has to navigate through the map and visit as many of these checkpoints
  as possible, in a given time interval(most often, 24 hours). This type of game
  has two branches: metrogaine(rogain through a city) and cyclogaine(rogain on
  bicycles).
  
  \item \textbf{'Tag' and 'Hot and Cold'} are classical children's games. Tag is
  a game in which one person is considered 'tagged'. That person passes the tag
  on by touching somebody else. The game is open ended. 'Hot and Cold' is a game
  in which one player has to find an item hidden by the others. During one's
  search for the item, the others help him by saying 'hot' when he is close and
  'cold' when he is getting away from the item. In a variation of the game the
  players can use degrees of hot and cold to further help the searcher find the
  item quicker.
    
\end{enumerate}

All these games will be found in the intermediary and final ideas for games or
game features that have come through the phase that we can call the
'Exploration Phase'.


\subsection{Multiplayer ideas for a tour app}

The whole project started with the intent of adding multiplayer functionality
and features to a tour application. A tour application is essentially for use by
a single person. The first feature that comes to mind when multiplayer is
considered is having the positions of all the other persons using the
application - so that one cannot go astray from the group without finding his
way back to the group. Then, a search was done for features that could add
dynamics to the multiuser approach. So came the idea of adding group quizzes and
small games. This meant that the team had to split and find various clues on
sidetracks of the main tour and solve short puzzles related to the landmarks
along the way. Then, this turned into more complicated concepts - such as adding
navigation: One of the members of the group would navigate to a goal in the same
manner of the 'Hot and Cold' game - either with the help of members of the group
of that of the mobile device.\newline

At some point, though, it has been determined that even if these small added
features to the game would be entertaining, they are not fit as research
subjects - they are already there to be implemented. Focus has moved to rather
full-on games that would implement something not thought of thus far. The idea
of augmenting the reality of sports such as orienteering and rogaining was
worked on and the result is one of the games, the 'Territory Takeover'. Radio
orienteering and racing have been combined in the 'Mine Game'. Features from
sports and video games have been added to what was proposed as 'The War Game'.
They will all be described below, along with some rough technical requirements
that have been considered.

\subsection{Ideas for Real Time Tactics games}

This paper focuses on the research and development of GPS-enabled Real
Time Tactics games. They are to be augmented reality games for single player or
multiplayer competitive 'free for all'/'skirmish' and 'team versus team'
games. This category of games offers opportunities to also enhance the
experience for all the previously existing types of GPS-enabled mobile games.

The games proposed for this project have been :
\begin{enumerate}	
	\item \textbf{Territory Takeover}
	\item \textbf{The War Game}
	\item \textbf{The Mine Game}
\end{enumerate}

1. The \textbf{Territory Takeover} game is a multiplayer, team versus team
competitive game. The players or game author define an area of play, which will
be divided into multiple divisions. Each division will be marked by a 'flag' (a
GPS marker). To capture the area division, a team must capture its flag. The
game ends when all flags have been captured and the winner is the team with most
captured flags. Each flag may be given a time that a player must spend next to
it in order to capture it. Once a flag (and implicitly the territory) is
captured, it remains so until the end of the game. The winner can be decided on
flag counting or, alternatively, each flag may receive a number of points,
according to the size of the territory marked by it and the difficulty of the
terrain. \newline

This game can be enhanced with the use of virtual tools or weapons. For the
purpose of this project, the following tools/weapons have been considered :
\begin{enumerate}
	
	\item The \textbf{Immobilizer} is an ability that can be used by each player to
	block an opponent from moving. The 'attacker' 'activates' the ability and a
	circle around him is drawn to show the range in which he can shoot. If an
	opponent enters the range area, the 'attacker' will select him on the map and
	shoot. The 'victim' will receive a notification that he is immobilized. A
	circle or rectangle will be defined around him and he will not be allowed to
	move outside of it for a given time, say 30 seconds or 1 minute. If he does, he
	gets disqualified and kicked out of the game. An alternate solution would be
	that the team loses points, for the case that this is the scoring methodology
	implemented.
	 
	\item \textbf{Demobilizer} is an ability that an immobilized player can use.
	For this project, it will only work on the person that uses the ability. The
	effect is that a person that is immobilized gets the waiting time halved.
	
\end{enumerate}
Both the abilities have a common cooldown timer. That means that if a player
immobilizes somebody and is immediately immoblized himself, he won't be able to
use the demobilizer because of the cooldown following the usage of the
immobilizer.\newline

For time and effort reasons, this game will not be implemented now, but kept as
future work: it can be added as an extra game type within the app in
development.\newline

2. \textbf{The War Game} is inspired by Real Time Strategy and Shooter games.
Two teams are formed. The area of play may be limited or unlimited. Each
player can choose between a number of characters. The first proposal has been
for four character types: Defender, Marine, Sniper and Heavy Trooper. Each of
the four characters has special abilities and characteristics :
\begin{enumerate}
	
	\item The \textbf{Defender} has the ability to generate shields for short
	periods of time. Members of the team can hide behind those shields for defence.
	The Defender may also act as a Medic and heal or revive members of the team. He
	has low health, long ability cooldowns and a sidearm with short range, small
	damage and fast cooldowns .
	
	\item The \textbf{Marine} has a weapon that can shoot a medium range with
	medium damage and fast cooldowns. He has medium health. 
	
	\item The \textbf{Sniper} has two weapons : the sniper rifle that can shoot at
	distant ranges and deal large damage to single targets and the sidearm, which
	is the same as the Defender's. His health is low, just like the Defender's. The
	sniper shot may penetrate the Defender's shield and cause reduced damage to one
	target.	
	
	\item The \textbf{Heavy Trooper} has three weapons : the bazooka, the sidearm
	and mines. The bazooka is a mid-range weapon with splash damage - it therefore
	can be fired against compact groups, such as the ones that might be hiding
	behind a shield. The bazooka cannot deal damage through the shield, but it may
	be shot next to it, causing damage from the side. The damage to each target
	varies from moderate to small, depending how far they are from the center of
	the 'projectile explosion'. The mines can be placed randomly on the map and
	their 'explosions' will not affect the members of the Heavy Trooper's team.
	Also they cannot be triggered by the members of his team. The damage dealt will
	be moderate, with splash damage, just like the bazooka projectile. The bazooka
	and the mines have long cooldowns, therefore the sidearm is added. The Heavy
	Trooper has high health.
	 
\end{enumerate}


3. \textbf{The Mine Game} is inspired by the classical game Minesweeper,
Warfinger GPS and running games such as the ones in Tourality. It is essentially
a single player game that can be played by many for score ranking. It may be
adapted in various ways directly into the 'War Game'. The purpose of the game is
that the player uses his phone as a mine detector and defuser and navigates
through a virtual mine field, racing against time to get from a start point to
an end point. A variant of this game could be of a team helping a
designated player navigate through the mine field without touching any
mines.\newline

Some intuitive advantages and drawbacks have been delineated for the game in
proposal:\newline

\textbf{Advantages}: It does not require specialized gear and setting, nor does
it need long amounts of time to be played. It can be enjoyed with a bunch of
friends on a sunny weekend afternoon.\newline
\textbf{Drawbacks}: It highly depends on GPS accuracy. This issue may affect
gameplay. It also does not feel as 'real' as real-life games, nor PC
games.\newline

The similarity of the construction of the three above-mentioned games would
allow us to use a common framework that will enable multiplayer interaction for
both 'free for all'/'skirmish' and 'team versus team' approaches and permit the
implementation of the 'Mine Game' along with the others. They would require a
server to centralize player information such as GPS data and the virtual
'health' attribute. Because the games proposed are fast-paced, they require
quick response times from the server, client and and the communication protocol
between them.

\subsection{Documentation on the Games}
Extensive documentation has been found from research done on education-oriented
GPS-games.\cite{pbarg1}\cite{pbarg2}\cite{pbarg3}\cite{pbarg4}\cite{pbarg5}\cite{pbarg6},
describing concepts and approaches in developing platforms and games to this
end\cite{pbarg3}, porting them to different locations\cite{pbarg4}, comparing
them and discussing their functionality\cite{pbarg6}\cite{pbarg1}\cite{pbarg2}
or discussing their effect and usefulness\cite{pbarg1}. Unfortunately for the
direction of this particular project, this documentation focuses exclusively on
GPS-enabled Adventure Games.\newline

The lack of documentation for the other games, except their websites has left
trying them out one by one as the only option to understand their components and
functionality. Additional information was retrieved from video descriptions,
examples and reviews of the games.\newline

During the preparation of this paper no games or documentation have been found
on GPS-enabled Real Time Strategy games. From the searches performed, no
location-aware mobile games have been found to fall in the genre of RTS.
However, one has been found in the genre of `Shooter Games` - MobileWar
- but, unfortunately, it did not work on Android, nor did it on iOS.\newline

The names of the games themselves have been chosen only as descriptions of their
nature and are prone to change during the steps of development and evaluation.

\subsection{Purpose of the games}
The games are proposed with group teambuilding and recreation in mind. They
require team strategy and cooperation. Territory Takeover allows for both team
versus team and free for all gameplay, allowing for both small and large groups
to play. The War Game is to be a fast paced game spanning a time interval in the
range of a few tens of minutes. Although it contains some Shooter elements -
such as the act of shooting itself - it focuses on strategy and tactics. Quick
reflexes might be required to shoot, but not to aim Where it lacks the realism
of simulations or the immersion of classical computer Real Time Strategy games,
it gains in the intensity of real-life experience and teamwork, without requiring
specialized equipment(Airsoft) or highly developed skills(computer Real Time
Strategy games). Therefore, the game that is about to be developed fills a niche
between casual and immersive games, bringing focus to social interaction. 


\subsection{Implementation}

Implementation will mean developing a server and a client application from
scratch, covering all the functionality needed for the main game - the so-called
'War Game' to work according to its description.

\subsubsection{Schedule}
This project, consisting of one server and one client application, was planned
to be developed in four steps:


\begin{enumerate}
  \item \textbf{Development} - During the first iteration of development, the
  most basic features of the game are to be implemented: basic server
  functionality that would allow the game to work, basic client functionality
  and the 'War Game' without all features.(2 months)
  
  \item \textbf{Testing and Evaluation} - During this phase, the game and its
  functionality will be livetested for feasibility and quality. New ideas will
  be sought and documented. Most importantly, player feedback will be
  gathered.(1 month)
  
  \item \textbf{Development} - During the second iteration of development, the
  'War Game' will be completed and, using its framework, the 'Territory
  Takeover' game will be implemented. Bugs will be removed and tweaks will be
  made to the framework and the game concepts to match the player feedback.(2
  months)
  
  \item \textbf{Evaluation and Completion} - During the second evaluation phase,
  both games will be tested for playability, player feedback will be gathered
  and the Dissertation Paper will be completed.(1 month)
    
\end{enumerate}


