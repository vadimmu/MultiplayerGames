
\section{Related work}

In this chapter, we will follow the games and concepts that have lead to the
assembly of the idea for the game developed on this project. We will define a
number of computer game genres, classify some of the most popular location-aware
games, game platforms and non-digital games that aided the concept. Then we will
present the evolution of the idea for this game prototype. The game concept
brought up here is part of a genre that has a long legacy among computer games,
but is not yet known to the location-aware games market. The genre is that of
`Real Time Strategy`(RTS)\cite{rts} games. Some other elements come from the
genre of 'Shooter' games\cite{shooter}. We will do that with a subgenre of RTS,
called Real Time Tactics(RTT)\cite{rttvsrts}. The game ''People With Guns'' is a
Real Time Tactics game with Shooter elements.\newline


\subsection{Introduction: Real Time Strategy games}

We will now go through the definitions of Real Time Strategy(RTS), Real Time
Tactics(RTT), Turn Based Strategy(TBS) and will highlight a comparizon between
RTS and TBS.

\subsubsection{What is Real Time Strategy?}

RTS games are defined as real-time (continuous time) competitive games, in which
several players fight against each other, either in a skirmish or team versus
team. The purpose is to defeat the opponents by taking real-time decisions on
managing resources and troops. The three main focus points of the RTS genre
are : \textbf{managing resource gathering}, \textbf{building a base to provide
troops} and \textbf{battle tactics}.\cite{rts}\newline

This project will focus on a subgenre of RTS, Real Time Tactics(RTT) - which,
instead of managing all three aspects of RTS, focuses on battle tactics.\newline

Because it is essentially a simplified approach to RTS\cite{rttvsrts}, RTT can
provide a good proof of concept and at the same time simple and enjoyable
playability. The advantage of having a RTT GPS-enabled game versus RTS is that
it greatly reduces the play time, requires less skills and has the potential of
being less stressful than RTS.\newline

\subsubsection{Real Time Strateg versus Turn Based Strategy}

Turn Based Strategy(TBS) games\cite{rtsvstbs}(such as chess and board games, for
example) allow the player to take his time and plan every move. As opposed to
the games in this genre, in RTS games the player has to deal with a much greater
deal of uncertainty and has to improvise on the go. The duration of a TBS game
is greater than that of a RTS game. This type of game is not in the scope of
this project, yet it deserves mentioning, as there are many such games for
mobile devices. Some notable implementations are the ones of Scotland
Yard(Ravensburger GMBH), Catan(Catan GMBH) and Monopoly(Hasbro, Inc.).\newline

In particular, there are two implementations for Scotland Yard that have been
conceptually interesting in the search. A comparison has been
made\cite{rttvsrts2} on whether it is better to port a board game in a
real-time(continuous-time) or turn-based mobile location-based game. These two
implementations are Mobilis XHunt(turn based) and MisterX Mobile(real-time). The
comparison has been made on 10 aspects: Fun, Smooth Progression, Dynamic
Gameplay, Ease of play, Stressless Gameplay, Communication, Strategy, Clear
Rules, Low Risk, Education. The conclusion was that there was no favorite
between the two, but it has been concluded that these 10 aspects have different
weights for the player and that Fun, Smooth Progression and Dynamic Gameplay
have a higher individual weight to players than the rest.\cite[p.5]{rttvsrts2}
\newline

Based on the knowledge gathered from the above-mentioned comparison, the aim of
this project will be to maximize the interactivity and dynamics of the RTS game.


\subsection{Location-Based Augmented Reality Games}

Although the mobile games market has evolved along with that of hardware
devices, the location-based game genre remains a largely untapped area. Many
interesting games have been attempted, but very few have received mass
adoption.\newline

In this chapter, we will define location-based games, define a number of
genres in which a number of location-based games that were studied fall
into. Then we will make the classification of the games and briefly describe the
evolution of ideas and their turn into the final concept of 'The War Game',
which in turn ended up to become ''People With Guns''.


\subsubsection{Location-based games}
Location-based games take advantage of the mobile devices' built-in receivers
for global positioning. They provide the user's location with an accuracy
ranging from a few to several dozen meters. \newline

location-based games came up long before this feature has become ubiquitous in mobile
phones and tablets. One of the first widespread location-based games is Geocaching.
It is composed of two parts:

\begin{enumerate}
  \item Placing physical caches at various locations that can be considered
  interesting or worth visiting and publishing their GPS positions (eg. on
  websites).
  \item Searching for various caches by using a GPS device.
\end{enumerate}

Along with the evolution of smartphones came that of the mobile games. GPS games
come in a lot of flavors, from location-based tours, adventure and investigation
games to various race games - single and multi player and massively multiplayer
online games.


\subsubsection{Types of location-based games}

There is a number of location-based games and game authoring tools that are
available for iOS and Android devices. The ones studied for this paper are :
ARIS, Tourality, Wherigo, conTAGion, Shadow Cities, SCVNGR, Please Stay Calm,
Parallel Mafia, Parallel Kingdom, Tripventure, Warfinger, Totem, Portal Hunt,
aMazing, Ingress, MobileWar, Mister X Mobile, Mobilis XHunt, Own This World,
MapAttack.\newline

For better understanding the classification done below, we will first define
each type of game:

\begin{enumerate}
  \item \textbf{Adventure/Investigation Game} - Game in which the player plays
  the role of a character in a story. The primordial characteristics of this
  genre is that it is focused on immersion in the story, puzzle solving and
  investigation, rather than on physical skills. Also, the tendency of this
  genre is towards single player experience, though occasionally multiplayer is
  also implemented (eg. the ARIS-based game 'Mentira').
  
  \item \textbf{Massively Multiplayer Online Game} - This type of game is
  designed to support large numbers of players (even in the number of millions
  in some cases) that play and interact in a persistent virtual world. This type
  of game allows both cooperative and competitive gameplay and is exclusively
  based on multiplayer. Subgenres include MMO Role Playing Games and
  MMO Shooters.
  
  \item \textbf{Casual Game} - Analogous to the MMO, the Casual Game is targeted
  at mass at a mass audience and can incorporate any type of game type. The
  particularity of this genre is that it aims at having simple rules, simple
  gameplay and requiring no specialized skills. 
  
  \item \textbf{Racing Game} - It's a genre defining a broad range of games. In
  the case of computer games, it describes mostly motorized vehicle racing. In
  the mobile context, it mostly describes racing on foot against time or through
  a number of checkpoints.
  
  \item \textbf{Shooter Game} - This one's a subgenre of action games. It
  focuses on first or third person experience, speed, aiming and reaction time.
  Usually the weapon is ranged, although close-combat weapons are included in
  most games.
  
\end{enumerate}

We will now classify the games/platforms based on the genres they best fall in:

\begin{enumerate}
  \item \textbf{Adventure/Investigation Games}
  		\begin{enumerate}
  	  		\item ARIS
  	  		\item Wherigo
  	  		\item Tripventure
  	  		\item Tidy City	  
  		\end{enumerate}
  		
  \item \textbf{Massively Multiplayer Online Games}
  		\begin{enumerate}
  	  		\item Shadow Cities
  	  		\item Please Stay Calm
  	  		\item conTAGion
  	  		\item Parallel Mafia
  	  		\item Parallel Kingdom
  	  		\item Portal Hunt
  	  		\item Ingress
  		\end{enumerate}
  		
  \item \textbf{Casual Games}
  		\begin{enumerate}
  	  		\item SCVNGR
  	  		\item Warfinger  	  		
  	  		\item aMazing 	
  	  		\item Own This World
  	  		\item MapAttack	  	  		
  		\end{enumerate}
  		  		  		
  \item \textbf{Racing Games}
  		\begin{enumerate}
  	  		\item Tourality  	  		 	  		  	  		
  		\end{enumerate}
  		
  \item \textbf{Shooter Games}
  		\begin{enumerate}
  	  		\item MobileWar	  		  	  		
  		\end{enumerate}  
\end{enumerate}

A special category is represented by Mister X Mobile and Mobilis XHunt, which
both bring a board game (Scotland Yard) to the mobile environment. While the
former adapts the board game to real-time gameplay(placing it closer to the
'Multiplayer Racing Games', with elements from 'Real Time Strategy Games'), the
latter falls in the definition of 'Turn Based Strategy' games.

\subsubsection{Physical games}

A number of games have played a significant role in the construction of the
ideas that led to the eventual assembly of the one that was decided for
implementation. These games originate in the years before the digital age or the
widespread of computers and GPS-enabled mobile devices. Some of these games have
already been implemented in mobile games: tag and racing games and geocaching.
We will mention all of them and briefly describe them and where they came in
conceptually:
\begin{enumerate}
  \item \textbf{Orienteering} is a sport in which a map and a compass are used
  to navigate through checkpoints, across unfamiliar terrain. Checkpoints
  colored in white and orange are placed at various features along the way. This
  sport has started as a training discipline for the military and later on
  expanded as a civilian sport.
  
  \item \textbf{Radio Orienteering} is worth mentioning separately, as it was in
  itself an important source of inspiration for the games proposed along the
  way. It features a radio for direction finding, besides the map and compass. 
  
  \item \textbf{Rogaine} is a cross-country navigation sport. As a difference
  from orienteering, where a path must be followed, in rogaining a larger number
  of checkpoints is given and there is a time limit. A team of two to five
  people has to navigate through the map and visit as many of these checkpoints
  as possible, in a given time interval(most often, 24 hours). This type of game
  has two branches: metrogaine(rogain through a city) and cyclogaine(rogain on
  bicycles).
  
  \item \textbf{'Tag' and 'Hot and Cold'} are classical children's games. Tag is
  a game in which one person is considered 'tagged'. That person passes the tag
  on by touching somebody else. The game is open ended. 'Hot and Cold' is a game
  in which one player has to find an item hidden by the others. During one's
  search for the item, the others help him by saying 'hot' when he is close and
  'cold' when he is getting away from the item. In a variation of the game the
  players can use degrees of hot and cold to further help the searcher find the
  item quicker.
    
\end{enumerate}

All these games will be found in the intermediary and final ideas for games or
game features that have come through the phase that we can call the
'Exploration Phase'.


\subsection{Multiplayer ideas for a tour app}

The whole project started with the intent of adding multiplayer functionality
and features to a tour application. A tour application is essentially for use by
a single person. The first necessity for such a concept is that people
performing the tour together see each other's positions on the map. Then, a
search was done for features that could add dynamics to the multiuser
experience. So came the idea of adding group quizzes and small games. This meant
that the team had to split and find various clues on sidetracks of the main tour
and solve short puzzles related to the landmarks along the way. Thereafter, this
turned into more complicated concepts - such as adding navigation: One of the
members of the group would navigate to a goal in the same manner of the 'Hot and
Cold' game - either with the help of members of the group of that of the mobile
device.\newline

At some point, though, it has been determined that even if these small added
features to the game would be entertaining, they are not fit as research
subjects - they are already there to be implemented. Focus has moved to full-on
games that would implement something not thought of thus far. The idea of
augmenting the reality of sports such as orienteering and rogaining has been
worked on and the result is one of the games, the 'Territory Takeover'. Radio
orienteering and racing have been combined in the 'Mine Game'. Features from
sports and video games have been added to what was proposed as 'The War Game'.
They will all be described below, along with some rough technical requirements
that have been considered.

\subsection{Documentation on the Games}
Extensive documentation has been found from research done on education-oriented
GPS-games.\cite{pbarg1} \cite{pbarg2} \cite{pbarg3} \cite{pbarg4} \cite{pbarg5}
\cite{pbarg6}, describing concepts and approaches in developing platforms and
games to this end\cite{pbarg3}, porting them to different
locations\cite{pbarg4}, comparing them and discussing their
functionality\cite{pbarg6} \cite{pbarg1} \cite{pbarg2} or discussing their
effect and usefulness\cite{pbarg1}. Although the genre of Adventure
Games has very little connection with that of RTS Games, it has directly
influenced the creation of ''People With Guns''. It has provided critical
information on how multiplayer features have been implemented for this genre and
how social interaction is provided through cooperative puzzle solving.


Unfortunately for the direction of this particular
project, this documentation focuses exclusively on GPS-enabled Adventure Games.\newline

The lack of documentation for the other games, except their websites has left
trying them out one by one as the only option to understand their components and
functionality. Additional information was retrieved from video descriptions,
examples and reviews of the games.\newline

During the preparation of this paper no games or documentation have been found
on GPS-enabled Real Time Strategy games. From the searches performed, no
location-aware mobile games have been found to fall in the genre of RTS.
However, one has been found in the genre of `Shooter Games` - MobileWar
- but, unfortunately, it did not work on Android, nor did it on iOS.\newline

The names of the three proposed games concepts have been chosen only as
descriptions of their nature and are prone to change during the steps of
development and evaluation.

