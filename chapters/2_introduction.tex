\section{Introduction}

Ever since the first operating system for handheld devices, the spreading of
smartphones has intensifed every year. In 2012, the number of smartphones in the
world has reached one billion, according to Bloomberg\cite{bloomberg}. \newline

The rapid expansion of mobile computing presents new challenges and
opportunities for both the user and the developer. The ease of use and the
presence of touchscreens, GPS receivers, gyroscopes, accelerometers and, since
recently, even barometers gives way to new approaches in developing
games.\newline

The mobile market for games has matured, as the quota of mobile games from the
total digital games market is increasing rapidly and concurrently with that of
mobile phones and tablets among all connected devices. A report of the
Entertainment Software Association shows that in 2012, 62\% of gamers played
games with others, whether they did it in person or online. 33\% of gamers
played socialgames\cite{esa}. According to a prediction by Forbes, the mobile
game industry will focus on multiplayer in the year 2013\cite{forbes}
and a study by Emarketer.com shows that from 2011 to 2012, the mobile
gaming revenues in the US have almost tripled and estimates that the
growth will reach in 2017 659\% of the level from 2011\cite{emarketer}.\newline

By searching and testing out the available games for Android and iOS, we can
conclude that the GPS-based game market is still making its first steps towards
the public, despide significant research and tryouts. Eighteen games that can
currently be found either in research or on the aforementioned markets have been
evaluated and tried out, the concepts and presentations of other mobile game
concepts, along with navigation sports and games that predate the computer have
been studied. The conclusion drawn has been that most of the GPS-based games in
existence focus more on virtuality than on reality. Although many are social
games or have social aspects to them, it is more likely to see two people
standing next to each other with their eyes on the phone than directly
interacting. Considering the Milgrams Virtuality-Reality Continuum, we could
rather say that the current GPS-based games are augmented virtuality games, in
which the player position is a reality feature added to a game of immersive
virtuality.\newline

The reason why this game can be considered important is that it fuses existing
video game genres with the mobile experience, into an original game with dynamic
and enjoyable gameplay - a process which is roughly based on(but strongly
inspired by) previous attempts by others. They will be described in the
process.\newline

The second chapter, 'Related work', will present the sports, games and game
creation frameworks that have inspired the concept behind ''People With Guns''.
\newline

The third chapter, 'The game concept' will detail the final concept of the game
that was developed and will briefly present a typical gameplay scenario.\newline

In the fourth chapter, 'Architecture' we will see how the game works and how its
components work together to bring forth the gaming experience proposed in the
second chapter.\newline

The fifth chapter presents the development timeline: the gradual development of
an idea into a game, step by step, covering both technical and design aspects in
their evolution.\newline

In the sixth chapter, the organization methodologies that have been used will be
presented, along with conclusions and suggestions based on the experience with
this project.\newline

The seventh chapter briefly describes ideas for the future stages of this game's
development\newline

Chapter eight draws the conclusion over the whole work\newline

At the end, six annexes have been added with a number of detailed insights on
technologies chosen, game descriptions and tutorials.