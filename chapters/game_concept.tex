\section{The game concept}

The game that has been chosen for this project is the one defined as 'The War
Game'. It promisses to be the game to offer the most panoptic multiplayer
experience, both as a serious and casual game.\newline

\subsection{Purpose}

The purpose of this project is to create a Real Time Tactics game with Shooter
elements that is fast paced and can be appealing to all ages and sexes. It is to
provide casual gameplay that can also be taken to more serious levels, according
to the skill of the players themselves. The main goal is to develop the game and
provide appealing functionality and dynamics. The second goal is to test whether
a broad spectrum of population may enjoy this game and if not, if it can be
modified and/or adapted to fit the needs of the ones left out.\newline

The 'War Game', which has later been named 'People With Guns', is a GPS-based
game in which two teams fight a last-man-standing battle. Each person gets to
choose between a number of character types in the game. How fast one moves in
the game equals how fast one runs in real life.\newline

The player can choose between four types of characters : Medic, Sniper, Scout,
Marine - each with their own specific weapons and health, fitting different
roles within the game.\newline

The rules of the game are simple: two times fight a virtual battle. One team's
purpose is to defeat the opposite team with the means given: each player's legs
for running and the virtual weapons and powerups for fighting. The interface
with the so-called 'weapons' and 'powerups' is layed out in the form of buttons
that are shown on the bottom of the screen. Each weapon or powerup has the
following attributes: range, cooldown, duration, damage. By default weapons have
instant effect(therefore no duration) and powerups have no damage(but they have
a duration) - the only exception is the 'Heal' ability. Both weapons and
cooldowns have a range - an area of effect for their use. If a target falls
within that area, the weapon or powerup of choice can be used. After the
activation of a weapon or powerup, it will be availavbe for use after a time
given by its 'cooldown' attribute. In the case of powerups('Invisibility' and
'Shield'), their time span during which they are in effect after activation will
be given by the 'duration' attribute. \newline

The players are to perform complex communication between each other verbally,
thus maintaining social contact as long as they are in each other's proximity.
For the case of players that are too far away from the rest of their team, a
number of preset messages are provided for quickly exchanging information
between them(such as asking for help, cover or for healing) without distracting
them from the gameplay.\newline

The purpose of this game is to add to the experience of a group of people,
without taking too much focus upon itself. Social bonding and team building are
the goal to be achieved through strategy, tactics and a fun, light game to bring
it all together.\newline

\subsection{Gameplay}

The gameplay will be presented as a typical scenario of interacting with the
application:

\begin{enumerate}
  \item First of all, in order to get in the game, the player must connect to
  the server. Because this version of the game was created for testing purposes
  only, the server can only host one game. Once there is somebody playing,
  nobody else can join the game. A few seconds after everyone has left the game
  (gone back to the Main Menu), the server will reset itself and accept clients
  once more.
  
  \item Second, once the player has connected to the server, he will be
  presented with the Game Lobby. This is where he joins a team, sets his
  nickname and chooses his character type.

  \item Third, after the player has finished setting up, he can mark himself as
  'Ready' to play the game. If all the players are Ready, the server will send a
  five second countdown and send the signal to enter the game - at which point,
  all connected clients will switch to the game screen and the players can
  start playing.
  
  \item Once in the game, the player's weapons will be enabled only when the
  teams satisfy the starting condition - for now, that means that the average
  positions of all the members of the two teams must be at least 150m apart and
  the members of each team must be at most 20m away from the center position of
  their team.
 
\end{enumerate}


The in-game controls are separated in six areas :

\begin{enumerate}
  \item \textbf{The bottom area}: the Weapons: For each weapon there is a
  separate button. Each button serves three purposes: If the player presses it,
  the weapon or powerup linked to that button will shoot. If the player presses
  a weapon button for a longer time, the range of the weapon will be drawn on
  the map. Once a weapon was successfully used(on a target within its range),
  the button will be temporarily disabled and will show the weapon cooldown
  countdown.
  
  \item \textbf{The down-right corner}: The target selection buttons: The player
  can select his targets (both friends and enemies) by clicking on their
  markers on the map. As an alternative, three buttons are there to help him:
  the Friends/Enemies toggle button, with which he can choose from which
  group the selection will be made: friends or enemies. Above and below the
  toggle button are the 'Next Target' and 'Prev Target'(Previous Target)
  buttons. By pressing the 'Next Target' button, he will select the next closest
  player(If there is one selected, the next closest one will be chosen. If
  nobody is selected, the closest one will be chosen.). The 'Prev Target' button
  gets the opposite: the previous farthest friend or foe is selected, according
  to the same logic as with 'Next Target'. Above the three buttons, he can find
  a text view which shows the distance to the selected target. By clicking a
  random empty area on the screen, he will deselect whichever player was
  selected. Having no one on the screen selected is equal to having oneself
  selected - this is necessary for using powerups on oneself.
  
  \item \textbf{The top-left corner}: The health and messages buttons: The
  player's health is shown in large font. Below it, he can find the message
  selection dropdown menu. This has been arranged so that he does not waste time
  typing, but instead send critical preset messages to his team, when verbal
  communication is not possible.
  
  \item \textbf{The top-right corner}: The current position button: If the
  player presses it, the map moves to have your position in the center.
  
  \item \textbf{The top-center area}: The powerup duration views: If the player
  enables a powerup or somebody uses a powerup on him(for now, this applies only
  to the 'Shield' powerup), he will see its remaining duration of the effect
  as a countdown on the top of the screen.
  
  \item \textbf{The area above the weapon buttons}: The information area:
  Whenever the player shoots, is shot, sends or receives a message a 'Toast'
  will appear with info. The 'Toast' is an Android-specific short message that
  appears on the screen for a short time.
  
\end{enumerate}

The strategy of the teams can vary and will be more succesful when they devise
one in which they help and back each other, by complementing their skills. That
is why this game can provide both easy, fun gameplay and serious and complicated
strategies, based on the intention of the people playing. Different combinations
of players in each team, according to the needs and style of each player are
possible - and they lead to ever-different approaches in the game.\newline

The game ends when one of the teams is eliminated. Each player's 'character' or
'profession' has a number of associated health points. When the number of health
points reaches zero, the player is eliminated from the game and shown a dialog
giving the options to either quit the game or spectate.\newline

