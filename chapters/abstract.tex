\section{Abstract}

Ever since the first operating system for handheld devices, the spreading of
smartphones has intensifed every year. In 2012, the number of smartphones in the
world has reached one billion, according to Bloomberg\cite{bloomberg} \newline

The rapid expansion of mobile computing presents new challenges and
opportunities for both the user and the developer. The ease of use and the
presence of touchscreens, GPS receivers, gyroscopes, accelerometers and, since
recently, even barometers gives way to new approaches in developing
games.\newline

The mobile market for games has matured, as the quota of mobile games from the
total digital games market is increasing rapidly and concurrently with that of
mobile phones and tablets among all connected devices. A report of the
Entertainment Software Association shows that in 2012, 62\% of gamers played
games with others, whether they did it in person or online. 33\% of gamers
played socialgames\cite{esa}. According to a prediction by Forbes, the mobile
game industry will focus on multiplayer in the year 2013\cite{forbes}
and a study by Emarketer.com shows that from 2011 to 2012, the mobile
gaming revenues in the US have almost tripled and estimates that the
growth will reach in 2017 659\% of the level from 2011\cite{emarketer}.\newline

By searching and testing out the available games for Android and iOS, we can
conclude that the GPS-based game market is still making its first steps towards
the public, despide significant research and tryouts. Eighteen games that can
currently be found either in research or on the aforementioned markets have been
evaluated and tried out, the concepts and presentations of other mobile game
concepts, along with navigation sports and games that predate the computer have
been studied. The conclusion drawn has been that most of the GPS-based games in
existence focus more on virtuality than on reality. Although many are social
games or have social aspects to them, it is more likely to see two people
standing next to each other with their eyes in the phone than directly
interacting. Considering the interval between reality and virtuality, we could
rather say that the current GPS-based games are augmented virtuality games, in
which the position is an extra element to an immersive game.\newline 

The purpose of this project is to create a niche game which provides augmented
reality, rather than augmented virtuality - a game that would bond a group of
people in social interaction by adding competition between two teams and
cooperation within the teams.\newline

This paper describes the research and development of \"Gun Run\", a
Real Time Tactics / Shooter hybrid game. \"Gun Run\" is a team versus team last
man standing virtual battle in which the players of the two teams run in real
life, using their mobile phones to communicate and interact with each other with
virtual 'weapons' and 'powerups'.\newline

We will first describe the steps taken until the idea was properly outlined -
the whole process of searching for inspiration and the construction of the idea.
Then, we will proceed to present the process of developing the application,
intertwined with a dynamic progression of the game concept and
mechanics.\newline

The reason why this game can be considered important is that it fuses existing
video game genres with the mobile experience, into an original game with dynamic
and enjoyable gameplay - a process which is roughly based on(but strongly
inspired by) previous attempts by others. They will be described in the
process.\newline

The first chapter will describe the initial proposal for the game, the steps
taken to construct the idea and some side ideas that are worthwhile adding to
the game over time, but which were temporarily set aside for time and effort
reasons.\newline

The second chapter will describe the progression of the concept and the
development of \"Gun Run\", the Real Time Tactics / Shooter
hybrid.\newline

The third chapter will cover design and gameplay aspects, measurements and
future work.\newline