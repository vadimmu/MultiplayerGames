%%%%%%%%%%%%%%%%%%%%%%%%%%%%%%%%%%%%%%%%%
% Laboratory Report LaTeX Template
%
% This template has been downloaded from:
% http://www.latextemplates.com
%
% Original header:
%
% This is a LaTeX version of the sample laboratory report
% from Virginia Tech's copyrighted 08-09 CHEM 1045/1046 lab manual.
% Reproduction of this one appendix section for academic purposes
% should fall under fair use.
%
%%%%%%%%%%%%%%%%%%%%%%%%%%%%%%%%%%%%%%%%%

%----------------------------------------------------------------------------------------
%	DOCUMENT CONFIGURATIONS
%----------------------------------------------------------------------------------------

\documentclass{article}

    \usepackage[a4paper,vmargin={30mm,30mm},hmargin={20mm,20mm}]{geometry}

\title{Multiplayer Mobile GPS Games \\ Expos� \\
Winter Semester 2012}
% Title

\author{Vadim Costache} % Author name

\begin{document}

\maketitle % Insert the title, author and date

\setlength\parindent{0pt} % Removes all indentation from paragraphs

\renewcommand{\labelenumi}{\alph{enumi}.} % Make numbering in the enumerate environment by letter rather than number (e.g. section 6)

\newcommand{\superscript}[1]{\ensuremath{^{\textrm{#1}}}}
\newcommand{\subscript}[1]{\ensuremath{_{\textrm{#1}}}}

%----------------------------------------------------------------------------------------
%	TABLE OF CONTENTS
%----------------------------------------------------------------------------------------
   
\begin{verbatim}



\end{verbatim}
     
\tableofcontents

\newpage

\section{Introduction: Location-Based Augmented Reality Games}

Ever since the first operating system for handheld devices, the spreading of
smartphones has intensified every year.\newline

The rapid expansion of mobile computing presents new challenges and
opportunities for both the user and the developer. The ease of use and the
presence of touchscreens, GPS receivers, gyroscopes, accelerometers and, since
recently, even barometers gives way to new approaches in developing
games.\newline

Although it is one of the oldest additions to the smartphone, the GPS-enabled
smartphone is still not ubiquitous. As of now, each company's flagship
and most of their mid-level smartphones are GPS-enabled. This makes way to the
propagation of GPS gaming. Although the concept is old, very few attempts have
been made in this direction and this branch of game development may be
considered to still be in one of its early stages of maturity.\newline

This paper is a proposal for extending the genre of mobile real-time multiplayer
location-based games.

\section{Location-based games}
Location-based games take advantage of the mobile devices' built-in receivers
for global positioning. They provide the user's location with an accuracy
ranging from a few to a couple dozen meters. Because the most mobile
devices in the world today rely on the Global Positioning System, we can
use the term GPS-games.\newline

GPS-games came up long before this feature has become ubiquitous in mobile
phones and tablets. The first widespread GPS-game is Geocaching. It is composed
of two parts: 
\begin{enumerate}
  \item Placing physical caches at various locations that can be considered
  interesting or worth visiting and publishing their GPS positions (eg. on
  websites).
  \item Searching for various caches by using a GPS device. 
\end{enumerate}

Along with the evolution of smartphones came that of the mobile games. GPS games
come in a lot of flavors, from GPS-based tours, adventure and investigation
games to various race games - single and multi player.


\section{Types of GPS-based games }

There is a number of GPS-enabled games and game authoring tools that are
available for iOS and Android devices. The ones studied for this paper are :
ARIS, Tourality, Wherigo, conTAGion, Shadow Cities, SCVNGR, Please
Stay Calm, Parallel Mafia, Parallel Kingdom, Tripventure, Warfinger, Tidy City,
Portal Hunt, aMazing, Ingress.
\newline

We will now cover the scope and the most interesting features of each of the
enummerated games/platforms:


\begin{enumerate}
  \item \textbf{Adventure/Investigation Games}
  		\begin{enumerate}
  	  		\item ARIS
  	  		\item Wherigo
  	  		\item Tripventure
  	  		\item Tidy City	  
  		\end{enumerate}
  		
  \item \textbf{Massively Multiplayer Online Games}
  		\begin{enumerate}
  	  		\item Shadow Cities
  	  		\item Please Stay Calm
  	  		\item conTAGion
  	  		\item Parallel Mafia
  	  		\item Parallel Kingdom
  	  		\item Portal Hunt
  	  		\item Ingress
  		\end{enumerate}
  		
  \item \textbf{Casual Games}
  		\begin{enumerate}
  	  		\item SCVNGR
  	  		\item Warfinger  	  		
  	  		\item aMazing 	  		  	  		
  		\end{enumerate}
  		
  \item \textbf{Multiplayer Racing Games}
  		\begin{enumerate}
  	  		\item Tourality  	  		 	  		  	  		
  		\end{enumerate}
\end{enumerate}


\section{Real Time Strategy Games}

This project proposal is for GPS-enabled augmented reality games for single
player or multiplayer competitive free for all and team versus team games where
virtual tools are used for enhancing the gameplay. This category of games offers
opportunities to also enhance the experience for all the previously existing ones.

For this project, the proposed games are :
\begin{enumerate}
	\item \textbf{Navigation Game}
	\item \textbf{Mine Detector}	
	\item \textbf{Territory Takeover}
	\item \textbf{The War Game}
\end{enumerate}


1. The \textbf{Navigation Game} is inspired by Radio Orienteering and the Hot
and Cold game played by kids - a branch of orienteering in which the participant uses a radio
in addition to the classical navigation methods. The radio is used for guidance
towards beacons placed through the trail. Two variants of the game are
proposed:
\begin{enumerate}
	
	\item \textbf{The Peer-to-Peer Navigation Game} - After reaching a checkpoint,
	the team has to designate one member to be the Seeker for a goal (checkpoint,
	geocache or something else). They vote. Once the vote is over, the entire team
	is confined to an area defined by a circle or rectangle around the checkpoint
	where the voting began. Only the Seeker is allowed to leave the confinement
	area. Also, the Seeker's map will not contain the destination marker. The team
	will have the destination marker enabled once the Seeker leaves the confinement
	area and will have to navigate him through text or voice chat. The goal may
	be as simple as reaching the marker or a quiz or a puzzle which the Seeker may
	solve with the help of the team. Once the Seeker finds the marker and retrieves
	the goal, the confinement restriction disappears and the team may proceed. An
	alternative can be that the Seeker must return to the position of the team in
	order to disable the confinement restriction. 
	
	\item \textbf{The Device-to-Peer Navigation Game} - After reaching a checkpoint
	in a single or multiplayer game, the player automatically becomes or is voted
	to be a Seeker and has to navigate by himself towards the goal. His map will
	not show the goal, but instead a gradient of color showing him how close or far
	he is from the location of the goal. 
	
\end{enumerate}
2. The \textbf{Mine Detector} is rather a single player game that has
potential to be extended to multiplayer use. The single player scenario is as
follows: an opposing player or a game author prepares an area with a number of
virtual 'mines'. The player who will cross the 'minefield'(the racer) has to get
from a marker(start marker) to another(end marker) in a race against time. One added
aspect is the player's virtual 'health': if the racer gets too close to a mine,
it 'explodes' and reduces his 'health'. If the racer's health reaches zero, he
loses. The racer will have an area around him that will act as 'mine detector'
coverage. The mines show up only if they fall inside the 'mine detector' area.
The player may defuse mines, but the process costs time. The strategy is up to
the player.\newline

Even if it is not in itself a multiplayer game, the Mine Detector can be added
as a subgame in multiplayer tour games.\newline

3. The \textbf{Territory Takeover} game is a multiplayer, team versus team
competitive game. The players or game author define an area of play, which will
be divided into multiple divisions. Each division will be marked by a 'flag' (a
GPS marker). To capture the area division, a team must capture its flag. The
game ends when all flags have been captured and the winner is the team with most
captured flags. Each flag may be given a time that a player must spend next to
it in order to capture it. Once a flag (and implicitly the territory) is
captured, it remains so until the end of the game. The winner can be decided on
flag counting or, alternatively, each flag may receive a number of points,
according to the size of the territory marked by it and the difficulty of the
terrain. \newline

This game can be enhanced with the use of virtual tools or weapons. For the
purpose of this project, the following tools/weapons have been considered :
\begin{enumerate}
	
	\item The \textbf{Immobilizer} is an ability that can be used by each player to
	block an opponent from moving. The 'attacker' 'activates' the ability and a
	circle around him is drawn to show the range in which he can shoot. If an
	opponent enters the range area, the 'attacker' will select him on the map and
	shoot. The 'victim' will receive a notification that he is immobilized. A
	circle or rectangle will be defined around him and he will not be allowed to
	move outside of it for a given time, say 30 seconds or 1 minute. If he does, he
	gets disqualified and kicked out of the game. An alternate solution would be
	that the team loses points, for the case that this is the scoring methodology
	implemented.
	 
	\item \textbf{Demobilizer} is an ability that an immobilized player can use.
	For this project, it will only work on the person that uses the ability. The
	effect is that a person that is immobilized gets the waiting time halved.
	
\end{enumerate}
Both the abilities have a common cooldown timer. That means that if a player
immobilizes somebody and is immediately immoblized himself, he won't be able to
use the demobilizer because of the cooldown following the usage of the
immobilizer.

4. The War Game is inspired by Real Time Strategy Games and Airsoft. Two
teams are formed. An area of play is delimited on the map. Each player can
choose between a number of characters. For the purpose of this project, four
characters are proposed: Defender, Marine, Sniper and Heavy Trooper. Each of the
four characters has special abilities and characteristics :
\begin{enumerate}
	
	\item The \textbf{Defender} has the ability to generate shields for short
	periods of time. Members of the team can hide behind those shields for defence.
	The Defender may also act as a Medic and heal or revive members of the team. He
	has low health, long ability cooldowns and a sidearm with short range, small
	damage and fast cooldowns .
	
	\item The \textbf{Marine} has a weapon that can shoot a medium range with
	medium damage and fast cooldowns. He has medium health. 
	
	\item The \textbf{Sniper} has two weapons : the sniper rifle that can shoot at
	distant ranges and deal large damage to single targets and the sidearm, which
	is the same as the Defender's. His health is low, just like the Defender's. The
	sniper shot may penetrate the Defender's shield and cause reduced damage to one
	target.	
	
	\item The \textbf{Heavy Trooper} has three weapons : the bazooka, the sidearm
	and mines. The bazooka is a mid-range weapon with splash damage - it therefore
	can be fired against compact groups, such as the ones that might be hiding
	behind a shield. The bazooka cannot deal damage through the shield, but it may
	be shot next to it, causing damage from the side. The damage to each target
	varies from moderate to small, depending how far they are from the center of
	the 'projectile explosion'. The mines can be placed randomly on the map and
	their 'explosions' will not affect the members of the Heavy Trooper's team.
	Also they cannot be triggered by the members of his team. The damage dealt will
	be moderate, with splash damage, just like the bazooka projectile. The bazooka
	and the mines have long cooldowns, therefore the sidearm is added. The Heavy
	Trooper has high health.
	 
\end{enumerate}

In the first iteration of this game, obstacles will be defined on the map by the
players or by the author of the game. In a further phase, a feature to treat
buildings and various object as obstacles automatically will be added.\newline

This game has some advantages and drawbacks, when compared to Airsoft:\newline
\textbf{Advantages}: It does not require specialized gear and setting, nor does
it need long amounts of time to be played. It can be enjoyed with a bunch of friends on
a sunny weekend afternoon.\newline
\textbf{Drawbacks}: It highly depends on GPS accuracy. This issue may affect
gameplay. It also does not feel as 'real' as Airsoft.\newline

\section{Application Frontend}
The following technologies have been considered for developing the frontend for
the application : Android, XCode and Multiplatoform APIs. The multiplatform APIs
offer the benefit of easy implementation for multiple platforms and operating
systems. The most visible disadvantage is the performance of such systems.
Because the games proposed are fast-paced, they require quick response times
from both the interface and the communication protocol. Therefore, the choice of
native app development makes more sense for this situation.\newline

The client frontend will be developed with Android, due to its accessibility and
versatility and the fact that it's ubiquitous. The development for iOS-based
devices is left as an option.

\section{Application Backend}
The games that are to be implemented will actively rely on communication with a
centralized server. This implies a constant Internet connection on each mobile
device. 
\subsection{Server technology}
The server will be implemented using NodeJS, due to its implementation of the
Event Loop and therefore it's little consumption of resources and bandwidth. 

\subsection{Communication Protocols}
During a game, active communication must be performed between the mobile devices
and the server in order to multicast all player positions and intentions to all
players. There are two alternatives to this approach :

\begin{enumerate}
	\item The game is created on the server, in a so-called `Game Room` which all
the players join. Once the game starts, communication will be
peer-to-peer.
	\item The game is created on the server and once the game starts, all players send
their positions to the server. The server is then responsible to multicasting
the positions to all the players. 
\end{enumerate}

In the general idea of communication protocol usage, the list of choices has
been narrowed down to three:

\begin{enumerate}
	\item WebRTC - A protocol that is to be part of the HTML 5 standard. It will be
	based on the RTP(Reliable Transport Protocol), which is the base for VoIP
	protocols and is itself based on UDP. It promises to be a very fast standard
	protocol, appropriate for audio and video streaming and massive multiplayer
	games. It is still in draft format and there are no official implementations
	for it. Implementing the protocol itself is outside the scope of this project.
	\item WebSockets - A protocol that is part of the HTML 5 standard. It is a low
	latency TCP-based protocol that promisses to replace http in several types of
	web applications.
	\item RTP - The protocol on which VoIP and WebRTC are based. It is based on UDP
	and it is designed for real-time streaming of data.
\end{enumerate}

From these three protocols, the current discussion is between WebSockets and
RTP. WebRTC is to be left as an option until its official release and
implementation for both NodeJS and Android.


\section{Authoring Tool}
For the purpose of this project, individual authoring for the types of games
will be attempted. The second iteration will be an attempt to create a universal
authoring tool for GPS games, from basic Tour Games, to Educational Games, to
the more complicated Battle Games. The purpose of this attempt is to allow
authors to creatively combine all the types of games and subgames mentioned here
in a variety of custom games. The tool should allow the authors to simply drag
and drop game elements into the game and add them actions and attributes that
are specific to the elements in all the games presented.\newline

It is to be emphasized that, because this is not a vital feature to the
development of the concepts of the Battle Games, the development of a universal
authoring tool is an optional feature.

\section{Documentation on the Games}
The lack of documentation on the games themselves and the concepts implemented
has left trying them out one by one as the only option to understand their
components and functionality. Additional information was retrieved from
video descriptions and examples of the games. An exception is made by the ARIS
platform, which comes with documentation about the design process and the games
proposed, and with open source code available. Unfortunately for this project,
ARIS focuses on educational games much more than on multiplayer
experience.\newline

Implementations for GPS-enabled, fast-paced multiplayer games for mobile devices
have not been found and therefore no documentation on them could be generated
or obtained. The purpose of this project is to create proofs of concept for the
GPS-enable games catalogued here as 'Battle Games'.




%----------------------------------------------------------------------------------------
%	BIBLIOGRAPHY
%----------------------------------------------------------------------------------------

\begin{thebibliography}{9}

\bibitem{Smith:2012qr}
Smith, J.~M. and Jones, A.~B. (2012).
\newblock {\em Chemistry}.
\newblock Publisher, City, 7th edition.

\end{thebibliography}




\end{document}



